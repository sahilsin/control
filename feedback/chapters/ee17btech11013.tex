\begin{enumerate}[label=\thesubsection.\arabic*.,ref=\thesubsection.\theenumi]
\numberwithin{equation}{enumi}

\item    A lead Compensator network includes a parallel combination of R and C in feed-forward path as shown in Fig. \ref{fig:ee17btech11013}. Find the gain of the compensator.
\begin{figure}[!ht]
    \begin{center}
    \resizebox{\columnwidth}{!}{% \begin{figure}
\begin{circuitikz}[american voltages]
\draw
  (0,0) to [short, *-](1,0) to [resistor, l=$R_1$] (5.5,0)
  (1, 0) -- (1, 1.5) to [capacitor, *-*, l=$C$] (5.5, 1.5) -- (5.5, 0) -- (6,0) to [resistor,, l=$R_2$] (6,-3) to [short, -*] (0, -3) 
  (6, 0) to [short, -*] (7, 0) 
  (6, -3) to [short, -*] (7, -3);
  \end{circuitikz}
% \end{figure}
}
    \end{center}
\caption{}
\label{fig:ee17btech11013}
\end{figure}
\\
\solution The transfer function  is
    \begin{align}
    T(s) &= \frac{V_o}{V_i}
\\
&= \frac{R_2}{\frac{\frac{1}{sC}R1}{\frac{1}{sC}+R1} + R2}
\\
&= \frac{s+\frac{1}{\tau}}{s+\frac{1}{\tau\alpha}}
\label{eq:ee17btech11013_ta_comp}
    \end{align}
%
where
    \begin{align}
\label{eq:ee17btech11013_ta}
\begin{split}
    \alpha &= \frac{R_2}{R_1 + R_2}\\
    \tau &= R_1C
\end{split}
    \end{align}
\item If the transfer function of the compensator  is 
\begin{align}
\label{eq:ee17btech11013_gain}
G_c(s) = \frac{s+2}{s+4}
\end{align}, find the value of RC.
%
\\
\solution     From \eqref{eq:ee17btech11013_gain}, \eqref{eq:ee17btech11013_ta_comp}
 and \eqref{eq:ee17btech11013_ta}


\begin{align}
    \tau = R_1C = 0.5
    \end{align}

\item    Find the value of RC for the compensator in Section \ref{sec:ee18btech11021}.
\\
\solution The compensator gain in this case is 
     \begin{align}
\label{eq:ee18btech11021_gain_ex}
    T(s) = \frac{(s+\frac{1}{3})}{s+1}
    \end{align}
 From \eqref{eq:ee18btech11021_gain_ex}, 
 \eqref{eq:ee17btech11013_ta_comp}
 and \eqref{eq:ee17btech11013_ta},
    \begin{align}
    \tau &= 3
\\
\implies    R_1C &= 3
    \end{align}

\end{enumerate}
