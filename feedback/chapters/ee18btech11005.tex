\begin{enumerate}[label=\thesubsection.\arabic*.,ref=\thesubsection.\theenumi]
\numberwithin{equation}{enumi}

\item Fig. \ref{fig:ee18btech11005_original_circuit} shows a  non-inverting op-amp configuration   with parameters described in Table \ref{table:ee18btech11005_Input_Table}.  Draw the equivalent control system.
\renewcommand{\thefigure}{\theenumi.\arabic{figure}}
%
\begin{figure}[!ht]
	\begin{center}
		
		\resizebox{\columnwidth}{!}{\input{./figs/ee18btech11005/original_circuit.tex}}
	\end{center}
\caption{}
\label{fig:ee18btech11005_original_circuit}
\end{figure}
%
\begin{table}[!ht]
\centering
\input{./tables/ee18btech11005/ee18btech11005_2.tex}
\caption{}
\label{table:ee18btech11005_Input_Table}
\end{table}
\\
\solution  See 	Fig. \ref{fig:ee18btech11005_equivalent_control_system}
\begin{figure}[!ht]
	\begin{center}
			\resizebox{\columnwidth}{!}{\input{./figs/ee18btech11005/equivalent_control_system.tex}}
	\end{center}
\caption{}
\label{fig:ee18btech11005_equivalent_control_system}
\end{figure}
\renewcommand{\thefigure}{\theenumi}

\item Draw the small signal model for Fig. \ref{fig:ee18btech11005_original_circuit}.
\\
\solution
The equivalent circuit of the amplifier is in Fig. \ref{fig:ee18btech11005_equivalent_circuit}
\begin{figure}[!ht]
	\begin{center}
		
		\resizebox{\columnwidth}{!}{\input{./figs/ee18btech11005/equivalent_circuit.tex}}
	\end{center}
\caption{}
\label{fig:ee18btech11005_equivalent_circuit}
\end{figure}

\item Assuming that the operational amplifier has infinite input resistance and zero output resistance, find  the {\em feedback factor} $H$.
\\
\solution From Fig. \ref{fig:ee18btech11005_equivalent_circuit},

\begin{align}
\label{eq:ee18btech11005_opamp_output}
V_0 &= GV_i
\\
 V_i &= V_s -V_f
\\
V_f &= \frac{R_1}{R_1+R_2}V_o
\end{align}
%
assuming that the current through $R_s$ is very small.  Thus, 
\begin{align}
H &=  \frac{V_f}{V_o} = \frac{R_1}{R_1+R_2}
\label{eq:ee18btech11005_H}
\end{align}
\item  Obtain the closed loop gain $T$ and summarize your results through a Table.
\\
\solution Table \ref{table:ee18btech11005_Output_Table} provides a summary.

\begin{align}
\label{eq:ee18btech11005_T}
T &=    \frac{V_0}{V_i}= \frac{G}{1+GH}
  \\
&= \frac{G\brak{R_1+R_2}}{\brak{R_1+R_2}+GR_1}
\end{align}
\begin{table}[!ht]
\centering
\input{./tables/ee18btech11005/ee18btech11005_1.tex}
\caption{}
\label{table:ee18btech11005_Output_Table}
\end{table}
%
\item Find the condition under which closed loop gain T is almost entirely determined by the feedback network.
\\
\solution If 

\begin{align}
 GH &\gg 1,
 \\
T &\approx \frac{1}{H}  = 1 + \frac{R_2}{R_1} 
\label{eq:ee18btech11005_T}
\end{align}
\item If 
\begin{align} 
G & = 10^4
\\
T &= 10,
\end{align}
find $H$.
%$\frac{R_2}{R_1}$.
\\
\solution From Table \ref{table:ee18btech11005_Output_Table}
\begin{align}
    T &=  \frac{G}{1+GH} = 10
\\
\implies  H &= 0.0999
%\frac{R_2}{R_1} &= 9.010
\end{align}
%\item What is the amount of feedback in decibels?
%\solution The value of F in decibals is given by 
%\begin{align}
%    F(dB) &= 20\log\brak{1+GH}\\
%F(dB) &= 60 dB
%\end{align}
\item {\em Gain Desensitivity:} If G decreases by 20\%,what is the corresponding decrease in T?  Comment.
\\
\solution From From Table \ref{table:ee18btech11005_Output_Table},
Given
\begin{align}
T &= \frac{G}{1+GH}
\\
\implies dT &= \frac{dG}{\brak{1+GH}^2}
\\
\implies \frac{dT}{T} &= \frac{1}{1+GH}\frac{dG}{G}
\end{align}
From the information available so far, 
\begin{align}
dG = 20\%, G = 10^4, H = 0.0999
\implies \frac{dT}{T} = 0.025\%
\end{align}
%
using the following code.
\begin{lstlisting}
codes/ee18btech11005/ee18btech11005.py
\end{lstlisting}
%
Thus the closed loop gain is almost invariant to a relatively large (20\%) variation in the open loop gain $G$.  This is known as gain desensitivity.
\end{enumerate}
