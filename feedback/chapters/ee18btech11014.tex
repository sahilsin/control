\begin{enumerate}[label=\thesection.\arabic*.,ref=\thesection.\theenumi]
\numberwithin{equation}{enumi}

\item Draw the equivalent control system for the feedback current amplifier shown in \ref{fig:ee18btech11014_Input}
\renewcommand{\thefigure}{\theenumi.\arabic{figure}}
\begin{figure}[h!]
	\begin{center}
		\resizebox{\columnwidth/1}{!}{\input{./figs/ee18btech11014/ee18btech11014_figa.tex}}
	\end{center}
	\caption{}
	\label{fig:ee18btech11014_Input}
\end{figure}
\\
\solution See Fig. 	\ref{fig:ee18btech11014_Control_System}.

\begin{figure}[ht!]
	\begin{center}
		\resizebox{\columnwidth}{!}{\input{./figs/ee18btech11014/ee18btech11014_figc.tex}}
	\end{center}
	\caption{}
	\label{fig:ee18btech11014_Control_System}
\end{figure}
\renewcommand{\thefigure}{\theenumi}
\item For the feedback current amplifier shown in \ref{fig:ee18btech11014_Input}, draw the Small-Signal Model. Neglect the Early effect in $Q_{1}$ and $Q_{2}$.\\
\solution See Fig. 	\ref{fig:ee18btech11014_Small_Signal}.

While drawing a Small-Signal Model, we ground all constant voltage sources and open all constant current sources. All Small-Signal paramters are obtained from DC-Analysis of the circuit. Neglecting Early effect, in Small-Signal Analysis a N-MOSFET is modelled as a Current Source with value of current equal to $g_{m}v_{gs}$ flowing from Drain to Source. Whereas a P-MOSFET is modelled as a Current Source with value of current equal to $g_{m}v_{sg}$ flowing from Source to Drain.
\begin{figure}[h!]
	\begin{center}
		\resizebox{\columnwidth/1}{!}{\input{./figs/ee18btech11014/ee18btech11014_figb.tex}}
	\end{center}
	\caption{Small Signal Model}
	\label{fig:ee18btech11014_Small_Signal}
\end{figure}

%------------------------------------------------------------------------%

%\item Describe how the given circuit is a Negetive Feedback Current Amplifier.\\
%\solution 
%For the feedback to be negative, $I_{f}$ must have the same polarity as $I_{s}$. To ascertain that this is the case, we assume an increase in $I_{s}$ and follow the change around the loop: An increase in $I_{s}$ causes $I_{i}$ to increase and the drain voltage of $Q_{1}$ will increase. Since this voltage is applied to the gate of the p-channel device $Q_{2}$ , its increase will cause $I_{o}$ , the drain current of $Q_{2}$, to decrease. Thus, the voltage across $R_{M}$ will decrease, which will cause $I_{f}$ to increase. This is the same polarity assumed for the initial change in
%$I_{s}$, verifying that the feedback is indeed negative.
%------------------------------------------------------------------------%
%
\item Write all the node/loop equations using KCL/KVL.
\\
\solution From Figs. 	\ref{fig:ee18btech11014_Input} and 	\ref{fig:ee18btech11014_Small_Signal},
%
\begin{align}
\label{eq:ee18btech11014_G_der1}
I_i &= \frac{v_B}{R_D}
\\
\label{eq:ee18btech11014_G_der2}
I_o &= -g_{m2}v_{B}
\\
\label{eq:ee18btech11014_H_der1}
v_{C} + v_{B} &= -I_oR_L
\\
\label{eq:ee18btech11014_H_der2}
v_C &= \brak{I_o - I_f}R_M
\end{align}
%

\item Find the Expression for the Open-Loop Gain $G$.
\\
\solution From \eqref{eq:ee18btech11014_G_der1} and \eqref{eq:ee18btech11014_G_der2},
%
\begin{align}
\label{eq:ee18btech11014_G}
G=\frac{I_{o}}{I_{i}} = -g_{m2}R_D
\end{align}
%In Small-Signal Model,
%\begin{align}
%v_{B} = I_{i}R_{D}\\
%v_{gs_{2}} = v_{B} = I_{i}R_{D}
%\end{align}

%In Small-Signal Analysis, P-MOSFET is modelled as a current source where current flows from Source to Drain. So, the value of current flowing from Source to Drain in P-MOSFET is,
%\begin{align}
%I_{o} =  -g_{m_{2}}v_{gs_{2}} = -g_{m_{2}}I_{i}R_{D}
%\end{align}
%So, the Open-Circuit Gain is
%\begin{align}
%\label{eq:ee18btech110014_G}
%G = \frac{I_{o}}{I_{i}} =  -g_{m_{2}}R_{D}
%\end{align}
%------------------------------------------------------------------------%

\item Find the Expression of the Feedback Factor $H$.
\\
\solution 
\begin{align}
H = \frac{I_{f}}{I_{o}},
\end{align}


From \eqref{eq:ee18btech11014_H_der1}
 and \eqref{eq:ee18btech11014_H_der2},
\begin{align}
\brak{I_o + I_f}R_M + v_{B} &= -I_oR_L
\\
\implies \brak{I_o + I_f}R_M + I_iR_D &=-I_oR_L 
\end{align}
from  \eqref{eq:ee18btech11014_G_der1}
\begin{align}
\implies \brak{1 + H}R_M + \frac{R_D}{G} &=-R_L 
\end{align}
dividing by  $I_o $
\begin{align}
\implies H &= -1 - \frac{1}{R_M} \brak{R_L -\frac{1}{g_{m2}}}
\end{align}
upon  substituting from  \eqref{eq:ee18btech11014_G} and simplifying.


$I_{o}$ is fed to a current divider formed by $R_{M}$ and $R_{F}$.
$R_{F}$ is a Large Resistance compared to Input resistance of Amplifier and so most of the current flows through it leaving a small current as input to Amplifier. Hence the voltage at point 'A' is very small and is considered, $v_{A} \simeq 0$. So $R_{F}$ and $R_{M}$ are parallel and Voltage Drop across them is same.
\begin{align}
(I_{o} + I_{f})R_{M} \simeq -I_{f}R_{F}\\
\frac{I_{f}}{I_{o}} \simeq -\frac{R_{M}}{R_{F}+R_{M}}
\end{align}
So, the Feedback Factor,
\begin{align}
\label{eq:ee18btech110014_H}
H \equiv \frac{I_{f}}{I_{o}} \simeq-\frac{R_{M}}{R_{F}+R_{M}}
\end{align}
%------------------------------------------------------------------------%
\item Find the Expression for the Closed-Loop Gain $T=\frac{I_{o}}{I_{s}}$. 
\\
\solution 
From \eqref{eq:ee18btech110014_G}
 and \eqref{eq:ee18btech110014_H},
\begin{align}
\label{eq:ee18btech110014_T}
T &= \frac{I_{o}}{I_{s}} = \frac{G}{1+GH}\\
&=-\frac{g_{m_{2}} R_{D}}{1+g_{m_{2}} R_{D} /\left(1+\frac{R_{F}}{R_{M}}\right)}
\\
\implies T &= -\frac{g_{m_{2}} R_{D}}{1+g_{m_{2}} R_{D} /\left(1+\frac{R_{F}}{R_{M}}\right)}
\end{align}
%------------------------------------------------------------------------%

\end{enumerate}
