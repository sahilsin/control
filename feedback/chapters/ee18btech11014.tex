\begin{enumerate}[label=\thesubsection.\arabic*.,ref=\thesubsection.\theenumi]
\numberwithin{equation}{enumi}
\numberwithin{figure}{enumi}

\item Draw the equivalent control system for the feedback current amplifier shown in \ref{fig:ee18btech11014_Input}
\renewcommand{\thefigure}{\theenumi.\arabic{figure}}
\begin{figure}[h!]
	\begin{center}
		\resizebox{\columnwidth/1}{!}{\input{./figs/ee18btech11014/ee18btech11014_figa.tex}}
	\end{center}
	\caption{}
	\label{fig:ee18btech11014_Input}
\end{figure}
\\
\solution See Fig. 	\ref{fig:ee18btech11014_Control_System}.

\begin{figure}[ht!]
	\begin{center}
		\resizebox{\columnwidth}{!}{\input{./figs/ee18btech11014/ee18btech11014_figc.tex}}
	\end{center}
	\caption{}
	\label{fig:ee18btech11014_Control_System}
\end{figure}
\renewcommand{\thefigure}{\theenumi}
\item For the feedback current amplifier shown in \ref{fig:ee18btech11014_Input}, draw the Small-Signal Model. Neglect the Early effect in $Q_{1}$ and $Q_{2}$.\\
\solution See Fig. 	\ref{fig:ee18btech11014_Small_Signal}.

While drawing a Small-Signal Model, we ground all constant voltage sources and open all constant current sources. All Small-Signal paramters are obtained from DC-Analysis of the circuit. Neglecting Early effect, in Small-Signal Analysis a N-MOSFET is modelled as a Current Source with value of current equal to $g_{m}v_{gs}$ flowing from Drain to Source. Whereas a P-MOSFET is modelled as a Current Source with value of current equal to $g_{m}v_{sg}$ flowing from Source to Drain.
\begin{figure}[h!]
	\begin{center}
		\resizebox{\columnwidth/1}{!}{\input{./figs/ee18btech11014/ee18btech11014_figb.tex}}
	\end{center}
	\caption{Small Signal Model}
	\label{fig:ee18btech11014_Small_Signal}
\end{figure}

%------------------------------------------------------------------------%

%\item Describe how the given circuit is a Negetive Feedback Current Amplifier.\\
%\solution 
%For the feedback to be negative, $I_{f}$ must have the same polarity as $I_{s}$. To ascertain that this is the case, we assume an increase in $I_{s}$ and follow the change around the loop: An increase in $I_{s}$ causes $I_{i}$ to increase and the drain voltage of $Q_{1}$ will increase. Since this voltage is applied to the gate of the p-channel device $Q_{2}$ , its increase will cause $I_{o}$ , the drain current of $Q_{2}$, to decrease. Thus, the voltage across $R_{M}$ will decrease, which will cause $I_{f}$ to increase. This is the same polarity assumed for the initial change in
%$I_{s}$, verifying that the feedback is indeed negative.
%------------------------------------------------------------------------%
%
\item Write all the node/loop equations using KCL/KVL.
\\
\solution From Figs. 	\ref{fig:ee18btech11014_Input} and 	\ref{fig:ee18btech11014_Small_Signal},
%
\begin{align}
\label{eq:ee18btech11014_G_der1}
I_i &= \frac{v_B}{R_D}
\\
\label{eq:ee18btech11014_G_der2}
I_o &= -g_{m_2}v_{B}
\\
\label{eq:ee18btech11014_H_der1}
v_{C} - v_{A} &= -I_fR_F
\\
\label{eq:ee18btech11014_H_der2}
v_C &= \brak{I_o + I_f}R_M
\\
I_i &= g_{m_1}v_{A}
\label{eq:ee18btech11014_vA}
\end{align}
%

\item Find the Expression for the Open-Loop Gain $G$.
\label{prob:ee18btech11014_G}
\\
\solution From \eqref{eq:ee18btech11014_G_der1} and \eqref{eq:ee18btech11014_G_der2},
%
\begin{align}
\label{eq:ee18btech11014_G}
G=\frac{I_{o}}{I_{i}} = -g_{m_2}R_D
\end{align}
%------------------------------------------------------------------------%

\item Find the Expression of the Feedback Factor $H$.
\\
\solution 
\begin{align}
H = \frac{I_{f}}{I_{o}},
\label{eq:ee18btech11014_Hdef}
\end{align}


From \eqref{eq:ee18btech11014_H_der1}
 and \eqref{eq:ee18btech11014_H_der2},
\begin{align}
\brak{I_o + I_f}R_M - v_{A} &= -I_fR_F
\\
\implies \brak{I_o + I_f}R_M + \frac{I_i}{g_{m_1}} &=-I_fR_F
\end{align}
from  \eqref{eq:ee18btech11014_vA}. Dividing by  $I_o $,%
\begin{align}
\implies \brak{1 + H}R_M + \frac{1}{g_{m_1}G} &=-HR_F 
\end{align}
%
upon substituting from \label{eq:ee18btech11014_G}
and \label{eq:ee18btech11014_Hdef}.  Simplifying further, we obtain
%
\begin{align}
\implies H &= \frac{\frac{1}{g_{m_1}g_{m_2}R_D} - R_M}{R_F+R_M}
\\
& \approx  -\frac{ R_M}{R_F+R_M}
\label{eq:ee18btech11014_H}
\end{align}
%
for $R_M \gg \frac{1}{g_{m_1}g_{m_2}R_D}$. 
%
%------------------------------------------------------------------------%
\item Find the Expression for the Closed-Loop Gain $T=\frac{I_{o}}{I_{s}}$. 
\\
\solution 
From \eqref{eq:ee18btech11014_G}
 and \eqref{eq:ee18btech11014_H}, 

\begin{align}
\label{eq:ee18btech11014_T}
T &= \frac{I_{o}}{I_{s}} = \frac{G}{1+GH}\\
&=-\frac{g_{m_{2}} R_{D}}{1+g_{m_{2}} R_{D} /\left(1+\frac{R_{F}}{R_{M}}\right)}
%\\
%\implies T &= -\frac{g_{m_{2}} R_{D}}{1+g_{m_{2}} R_{D} /\left(1+\frac{R_{F}}{R_{M}}\right)}
\end{align}

\end{enumerate}
