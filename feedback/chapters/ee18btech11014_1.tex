\begin{enumerate}[label=\thesubsection.\arabic*.,ref=\thesubsection.\theenumi]
\numberwithin{equation}{enumi}
\numberwithin{figure}{enumi}

%------------------------------------------------------------------------%
\item Draw the Block Diagram and Circuit Diagram for $H$.\\
\solution The Block Diagram  is available in Fig. 	\ref{fig:ee18btech11014_Feedback_Block}

\renewcommand{\thefigure}{\theenumi.\arabic{figure}}
\begin{figure}[ht!]
	\begin{center}
		\resizebox{\columnwidth}{!}{\input{./figs/ee18btech11014/ee18btech11014_figf.tex}}
	\end{center}
	\caption{Feedback Block Diagram}
	\label{fig:ee18btech11014_Feedback_Block}
\end{figure}
and the corresponding circuit diagram in Fig.  \ref{fig:ee18btech11014_Feedback_Network}
\begin{figure}[ht!]
	\begin{center}
		\resizebox{\columnwidth/2}{!}{\input{./figs/ee18btech11014/ee18btech11014_figd.tex}}
	\end{center}
	\caption{Feedback Network}
	\label{fig:ee18btech11014_Feedback_Network}
\end{figure}
\renewcommand{\thefigure}{\theenumi}
\item Find $H$ from Fig.  \ref{fig:ee18btech11014_Feedback_Network}.
\\
\solution Using current division,
\begin{align}
\frac{I_{f}}{I_{o}} &= -\frac{R_{M}}{R_{F}+R_{M}}
\\
\implies H &= -\frac{R_{M}}{R_{F}+R_{M}}
\end{align}

%------------------------------------------------------------------------%

\item Find $R_{11}$ and $R_{22}$  of Feedback Network where $R_{11}$ is input resistance through Port-1 and $R_{22}$ is Input Resistance through Port-2.\\
\solution $R_{11}$ is calculated by opening the current source at  Port-2.  Hence, 
\begin{align}
R_{11} = R_{F} + R_{M}
\end{align}
While calculating $R_{22}$, Port-1 should be shorted. Hence, 
\begin{align}
R_{22} &= R_{F} || R_{M}\\
 &= \frac{R_{F}R_{M}}{R_{F}+R_{M}}
\end{align}
%------------------------------------------------------------------------%

\item Draw the block diagram and circuit diagram for calculating $G$.\\
\solution  See Figs. 		\ref{fig:ee18btech11014_OpenLoop_Block}
 and \ref{fig:ee18btech11014_OpenLoop_Network}
%
\renewcommand{\thefigure}{\theenumi.\arabic{figure}}
\begin{figure}[ht!]
	\begin{center}
		\resizebox{\columnwidth}{!}{\input{./figs/ee18btech11014/ee18btech11014_figg.tex}}
	\end{center}
	\caption{Open-Loop Block Diagram}
	\label{fig:ee18btech11014_OpenLoop_Block}
\end{figure}
\begin{figure}[ht!]
	\begin{center}
		\resizebox{\columnwidth}{!}{\input{./figs/ee18btech11014/ee18btech11014_fige.tex}}
	\end{center}
	\caption{Open-Loop Network}
	\label{fig:ee18btech11014_OpenLoop_Network}
\end{figure}
\renewcommand{\thefigure}{\theenumi}
\item Find $G$.
\\
\solution The analysis is the same as Problem \ref{prob:ee18btech11014_G}.


%By KVL and KCL,
%\begin{align}
%v_{B} = I_{i}R_{D}\\
%v_{gs_{2}} = v_{B} = I_{i}R_{D}\\
%I_{o} =  -g_{m_{2}}v_{gs_{2}} = -g_{m_{2}}I_{i}R_{D}\\
%\frac{I_{o}}{I_{i}} = -g_{m_{2}}R_{D}
%\end{align}
%
%So, Open-Loop Gain is
%\begin{align}
%G = \frac{I_{o}}{I_{i}} =  -g_{m_{2}}R_{D}
%\end{align}
%
%The Block Diagram of Open-Loop Network is

\end{enumerate}
