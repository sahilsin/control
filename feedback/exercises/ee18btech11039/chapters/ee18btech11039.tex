A series-shunt feedback amplifier employs a basic amplifier with parameters listed in 
Table \ref{table:ee18btech11039_tab1}.
%
\begin{table}[!ht]
\centering
\input{./tables/ee18btech11039/ee18btech11039_tab1.tex}
\caption{}
\label{table:ee18btech11039_tab1}
\end{table}
%
% aailableinput and output resistances each of $2K\ohm$ and
%    gain G = 1000V/V. The feedback factor H = 0.1V/V. 
Find the input resistance $R_{if}$, output resistance $R_{of}$ and gain 
    of the closed-loop amplifier.  
%\begin{enumerate}[label=\thesubsection.\arabic*.,ref=\thesubsection.\theenumi]
\begin{enumerate}[label=\arabic*.,ref=\theenumi]
    \numberwithin{equation}{enumi}
    \numberwithin{figure}{enumi}

\item Represent the parameters in Table \ref{table:ee18btech11039_tab1} through a circuit.
\\
\solution See Fig. \ref{fig:ee18btech11039_fig1}. 
\begin{figure}[!h]
		\resizebox{\columnwidth}{!}{\input{./figs/ee18btech11039/fig1.tex}}
\caption{Ideal structure}
\label{fig:ee18btech11039_fig1}
\end{figure}

 \item  Design the $H$ circuit.\\
\solution 
From fig:\ref{fig:ee18btech11039_fig5}
\begin{align}
H = \frac{V_f}{V_o} = \frac{R_1}{R_1 + R_2} = 0.1
\end{align}
\begin{figure}[!h]
		\resizebox{\columnwidth}{!}{\begin{circuitikz}[american]
    \ctikzset{tripoles/mos style/arrows}
    \draw (0, -2)to[open, v^<=$V_f$, *-*](0, 0)to[short, i = \mbox{$I = 0$}](2, 0)--(2.5, 0)to[resistor, l = $R_2$](5, 0)to[open, v^>=$V_o$, *-*](5, -2)--(2.5, -2)node[ground]{};
    \draw (2.5, 0)to[resistor, l = $R_1$, *-*](2.5, -2)--(0, -2);
    \end{circuitikz}}
\caption{H circuit}
\label{fig:ee18btech11039_fig5}
\end{figure}
%
where $R_1$ and $R_2$ are listed in Table \ref{table:ee18btech11039_tab2}.
%
\begin{table}[!h]
\centering
\input{./tables/ee18btech11039/ee18btech11039_tab2.tex}
\caption{Parameter values}
\label{table:ee18btech11039_tab2}
\end{table}
%
\item Find the closed loop gain $T$ and design the equivalent circuit.
\\
\solution See Fig. \ref{fig:ee18btech11039_fig3} for the equivalent circuit for $T$.
\begin{align}
    T &= \frac{G}{1+GH}
\\
      &= 9.9
\end{align}
%
\begin{figure}[!h]
		\resizebox{\columnwidth}{!}{\input{./figs/ee18btech11039/fig3.tex}}
\caption{Amplifier design}
\label{fig:ee18btech11039_fig3}
\end{figure}

\item Show $R_{if}$ and $R_{of}$ for the closed loop circuit and compute them.
\\
\solution See Fig. \ref{fig:ee18btech11039_fig2}.
%
\begin{align}
    R_{if} &= \brak{1+GH} R_i
    &= 202K\ohm
\end{align}

\begin{align}
    R_{of} &= \frac{R_o}{1+GH}
           &= 19.802\ohm
\end{align}
\begin{figure}[!h]
		\resizebox{\columnwidth}{!}{\input{./figs/ee18btech11039/fig2.tex}}
\caption{Equivalent circuit}
\label{fig:ee18btech11039_fig2}
\end{figure}



%\begin{figure}[!h]
%		\resizebox{\columnwidth}{!}{\input{./figs/ee18btech11039/fig4.tex}}
%\caption{G circuit}
%\label{fig:ee18btech11039_fig4}
%\end{figure}
%
%
%
%\textbf{Circuit design: }
%From fig:\ref{fig:ee18btech11039_fig4}
%\begin{align}
%G = \mu \frac{R_L || \brak{R_1 + R_2}}{\sbrak{R_L || \brak{R_1 + R_2}}+r_o} \frac{R_{id}}{R_{id}+R_s+\brak{R_1||R_2}} = 1000
%\end{align}
%
%Open-loop input resistance,
%\begin{align}
%     R_i =  R_s + R_{id} + \brak{R_1 || R_2} = 2K\ohm
%\end{align}
%
%
%Open-loop output resistance,
%\begin{align}
%      R_o = r_o || R_L || \brak{R_2 + R_1} = 2K\ohm
%\end{align}
%
%In fig.\ref{fig:ee18btech11039_fig6}, 
%\begin{align}
%    V_{s}\brak{t} = \sin\brak{2000\pi t} \\
%    \Rightarrow V_{s}\brak{s} = \frac{2000\pi}{s^2 + \brak{2000\pi}^2}
%\end{align}
%\begin{align}
%    V_{o}\brak{t} = 9.9\sin\brak{2000\pi t}\\
%    \Rightarrow V_{o}\brak{s} = \frac{9.9\brak{2000\pi}}{s^2 + \brak{2000\pi}^2}
%\end{align}
%\begin{align}
%    T = \left|\frac{V_o\brak{s}}{V_{s}\brak{s}}\right| = 9.9
%\end{align}
\item Verify the amplifier gain for a sinusoidal input of 1kHz through spice.
\\
\solution See Fig. \ref{fig:ee18btech11039_fig6} generated using
\begin{lstlisting}
/codes/ee18btech11039/ee18btech11039_spice1.py
\end{lstlisting}
%
\begin{figure}[!h]
		\includegraphics[width=\columnwidth]{./figs/ee18btech11039/spice_1.eps}
\caption{Time domain output of the simulation}
\label{fig:ee18btech11039_fig6}
\end{figure}

%The plot in fig:\ref{fig:ee18btech11039_fig7} verifies that the value of T for the circuit is constant for various values of frequencies, $f \in \brak{1, 10^8}$.\\

%The output in fig:\ref{fig:ee18btech11039_fig6} 
%\begin{figure}[!h]
%		\includegraphics[width=\columnwidth]{./figs/ee18btech11039/spice_2.eps}
%\caption{AC analysis, f = 1Hz to 100MHz}
%\label{fig:ee18btech11039_fig7}
%\end{figure}
% 
%The output in fig:\ref{fig:ee18btech11039_fig7} is obtained by running the following program-
%\begin{lstlisting}
%/codes/ee18btech11039/ee18btech11039_spice2.py
%\end{lstlisting}
\end{enumerate}
