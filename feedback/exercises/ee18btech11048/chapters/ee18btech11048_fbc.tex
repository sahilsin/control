Figure \ref{fig:ee18btech11048_original_circuit} shows a feedback transconductance
amplifier implemented using an op amp with open-loop gain $\mu$, a very large input resistance, and an output resistance $r_{o}$.
The output current $I_{o}$ that is delivered to the load resistance $R_{L}$
is sensed by the feedback network composed of the three
resistances $R_{M}$, $R_{1}$, and $R_{2}$, and a proportional voltage$V_{f}$
is fed back to the negative-input terminal of the op amp.
Find G,H and T. If the loop gain is large, find an approximate expression for T
and state precisely the condition for which this applies.
\begin{figure}[!ht]
	\begin{center}
		\resizebox{\columnwidth}{!}{\begin{circuitikz}
\ctikzset{bipoles/length=1cm}
\draw (0,0) node[op amp,yscale=-1.0] {}
(opamp.+) (-0.85, +0.35) -- (-3, +0.35) to [V=$V_s$] (-3,-1) to (-3,-1) node[ground]{}
(opamp.-) (-0.85, -0.35) -- (-0.85, -2) to [R=$R_1$] (-3,-2) to (-3,-2) node[ground]{}
(opamp.-) (-0.85, -0.35) -- (-0.85, -2) to [R=$R_2$] (0.85,-2) to (0.85,-2) 
(opamp.out) (0.85,0) to (0.85,-0.5) to [R=$R_L$] (0.85,-2) to (0.85,-2) to [R=$R_M$] (0.85,-3.5) to (0.85,-3.5)node[ground]{}

node at(-1.9,-1.7){$V_f$}
node at(-1.5,-1.7){$+$}
node at(-2.3,-1.7){$-$}
node at(1.1, -2) {$A$}
node at(1.1, 0) {$V_o$}
node at(-0.8,-0.06){$V_i$}
node at(-0.8,+0.25){$+$}
node at(-0.8,-0.3){$-$}
node at(+1.3,-0.6){$I_o$}

[arrows=<-,line width=.4pt](1,-.7)--(1,-.4);
\end{circuitikz}
}
	\end{center}
\caption{}
\label{fig:ee18btech11048_original_circuit}
\end{figure}
The parameters given are shown in the TABLE.\ref{table:ee18btech11048_ Table1}
\begin{table}[!ht]
\centering
\input{./tables/ee18btech11048_table1.tex}
\caption{1}
\label{table:ee18btech11048_ Table1}
\end{table}

\begin{enumerate}[label=\arabic*.,ref=\theenumi]
%\begin{enumerate}[label=\thesubsection.\arabic*.,ref=\thesubsection.\theenumi]
\numberwithin{equation}{enumi}

\item Draw the block diagram and the equivalent circuit for Fig. \ref{fig:ee18btech11048_original_circuit}\\
\solution
The equivalent circuit of the amplifier is in Fig. \ref{fig:ee18btech11048_ss_circuit}

\begin{figure}[!ht]
	\begin{center}
		\resizebox{\columnwidth}{!}{\usetikzlibrary{decorations.markings}
\begin{circuitikz}
\ctikzset{bipoles/length=1cm}

\draw 
(0, 0) to[V=$V_s$] (0,-1.5) to (0,-1.5) node[ground]{}
(0,0) -- (0,1)-- (1.5,1)  node at(1.8,1){$+$}
(1.5,-1.25)  node at(1.7,-1.25){$-$} 
(1.5,-1.25) -- (1,-1.25) -- (1,-1.75) to[R=$R_1$] (1,-2.75) --(1,-3) node[ground]{}
(1,-1.5) to[R=$R_2$] (5.5,-1.5){}
(3.5,1) to[R=$r_o$] (5.5,1){}
(5.5,-1.5) to[R=$R_L$] (5.5,1){}
(3.5,1) --(3.5,1) to[V=$\mu V_i$] (3.5,-0.5) node[ground]{}
(5.5,-1.5) to[R=$R_M$] (5.5,-2.75) --(5.5,-3) node[ground]{}
(5.5,1) --(6,1) node at(6.3,1){$V_0$}
node at(1.8,-0.3) {$V_i$}
node at(0.6,-1.75){$+$}
node at(0.6,-3){$-$}
node at(0.6,-2.5){$V_f$}
node at(6,0.3){$I_o$}
[arrows=<-,line width=.4pt](5.7,0.2)--(5.7,0.5);
\end{circuitikz}
}
	\end{center}
\caption{}
\label{fig:ee18btech11048_ss_circuit}
\end{figure}


\item Draw the block diagram and equivalent ciruit for $H$.
\\
\solution See Fig. \ref{} and \ref{}.
\item Find  $H$.
\\
\solution From Fig. \ref{},

\begin{align}
H &= \frac{V_{f}}{I_{o}} 
\\
&= \frac{R_1R_M}{R_1+R_2+R_M}
\label{eq:ee18btech11048_H}
\end{align}
%
\item Find  $G$.
\\
\solution From Fig. \ref{fig:ee18btech11048_ss_circuit},
\begin{align}
G &= \frac{I_{o}}{V_{i}} \label{eq:ee18btech11048_G}\\
&= \mu
\end{align}
\item  Find $T$.
\\
\solution
\begin{align}
T &= \frac{G}{1+GH} \label{eq:ee18btech11048_T}
\\
&= \frac{\mu \brak{R_1+R_2+R_M}}{R_1+R_2+R_M+ \mu R_1R_M}
\\
 &\approx \frac{1}{H}  = \frac{R_1+R_2+R_M}{R_1R_M} 
\end{align}

\item Summarize your results in a table.
\\
\solution See Table \ref{table:ee18btech11048_ Input_Table}


\begin{table}[!ht]
\centering
\input{./tables/ee18btech11048_tables2.tex}
\caption{}
\label{table:ee18btech11048_ Input_Table}
\end{table}


\item Find $V_o$ for the parameters given in  Table  \ref{table:ee18btech11048_ Example_Table}.
\begin{table}[!ht]
\centering
\input{./tables/ee18btech11048_tables3.tex}
\caption{}
\label{table:ee18btech11048_ Example_Table}
\end{table}
\solution  The following code computes the value of $V_o$ using the fact that
\begin{align}
V_o = \frac{V_s}{H}
\end{align}
%From Fig\ref{fig:ee18btech11048_original_circuit}
%\begin{align}
%V_o - V_A &= I_o R_L \\
%V_A &= I_o\brak{R_M\parallel\brak{R_1+R_2}}\\
%\implies V_o &= I_o\brak{R_L+\brak{R_M\parallel\brak{R_1+R_2}}}
%\end{align}
%Dividing both sides by $V_s$
%\begin{align}
%\frac{V_o}{V_s} &= \frac{I_o}{V_s}\brak{R_L+\brak{R_M\parallel\brak{R_1+R_2}}} \label{eq:ee18btech11048_eee}
%\end{align}
%From equation \ref{eq:ee18btech11048_To} and \ref{eq:ee18btech11048_eee}
%\begin{align}
% \frac{V_o}{V_s} &= T\brak{R_L+\brak{R_M\parallel\brak{R_1+R_2}}} \label{eq:ee18btech11048_final}
%\end{align}
%From values in table\ref{table:ee18btech11048_ Example_Table} 
%\begin{align}
% H&= \frac{\brak{1000}\brak{1000}}{1000+1000+1000}\\
% \implies H&= \frac{1000}{3}
%\end{align}
%For an op amp:
%\begin{align}
% G \in \brak{20000,200000}
%\end{align}
%So, from equation \ref{eq:ee18btech11048_greater} and Table\ref{table:ee18btech11048_ Example_Table}
%\begin{align}
%T &\approx \frac{1}{H}\\
% T &= \frac{3}{1000}
%\end{align}
%Hence,
%\begin{align}
%V_o &= 5V_s\\
%\implies V_o &= 5V
%\end{align}
\item Verify your result through spice.
\\
\solution
\end{enumerate}
