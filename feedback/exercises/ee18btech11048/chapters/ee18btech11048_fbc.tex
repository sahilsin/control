Figure \ref{fig:ee18btech11048_original_circuit} shows a feedback transconductance
amplifier implemented using an op amp with open-loop gain $\mu$, a very large input resistance, and an output resistance $r_{o}$.
The output current $I_{o}$ that is delivered to the load resistance $R_{L}$
is sensed by the feedback network composed of the three
resistances $R_{M}$, $R_{1}$, and $R_{2}$, and a proportional voltage$V_{f}$
is fed back to the negative-input terminal of the op amp.
Find G,H and T. If the loop gain is large, find an approximate expression for T
and state precisely the condition for which this applies.
\begin{figure}[!ht]
	\begin{center}
		\resizebox{\columnwidth}{!}{\begin{circuitikz}
\ctikzset{bipoles/length=1cm}
\draw (0,0) node[op amp,yscale=-1.0] {}
(opamp.+) (-0.85, +0.35) -- (-3, +0.35) to [V=$V_s$] (-3,-1) to (-3,-1) node[ground]{}
(opamp.-) (-0.85, -0.35) -- (-0.85, -2) to [R=$R_1$] (-3,-2) to (-3,-2) node[ground]{}
(opamp.-) (-0.85, -0.35) -- (-0.85, -2) to [R=$R_2$] (0.85,-2) to (0.85,-2) 
(opamp.out) (0.85,0) to (0.85,-0.5) to [R=$R_L$] (0.85,-2) to (0.85,-2) to [R=$R_M$] (0.85,-3.5) to (0.85,-3.5)node[ground]{}

node at(-1.9,-1.7){$V_f$}
node at(-1.5,-1.7){$+$}
node at(-2.3,-1.7){$-$}
node at(1.1, -2) {$A$}
node at(1.1, 0) {$V_o$}
node at(-0.8,-0.06){$V_i$}
node at(-0.8,+0.25){$+$}
node at(-0.8,-0.3){$-$}
node at(+1.3,-0.6){$I_o$}

[arrows=<-,line width=.4pt](1,-.7)--(1,-.4);
\end{circuitikz}
}
	\end{center}
\caption{}
\label{fig:ee18btech11048_original_circuit}
\end{figure}
The parameters given are shown in the TABLE.\ref{table:ee18btech11048_ Table1}
\begin{table}[!ht]
\centering
\input{./tables/ee18btech11048_table1.tex}
\caption{1}
\label{table:ee18btech11048_ Table1}
\end{table}

\begin{enumerate}[label=\arabic*.,ref=\theenumi]
%\begin{enumerate}[label=\thesubsection.\arabic*.,ref=\thesubsection.\theenumi]
\numberwithin{equation}{enumi}

\item Draw the block diagram and the equivalent circuit for Fig. \ref{fig:ee18btech11048_original_circuit}\\
\solution
The equivalent circuit of the amplifier is in Fig. \ref{fig:ee18btech11048_ss_circuit}

\begin{figure}[!ht]
	\begin{center}
		\resizebox{\columnwidth}{!}{\usetikzlibrary{decorations.markings}
\begin{circuitikz}
\ctikzset{bipoles/length=1cm}

\draw 
(0, 0) to[V=$V_s$] (0,-1.5) to (0,-1.5) node[ground]{}
(0,0) -- (0,1)-- (1.5,1)  node at(1.8,1){$+$}
(1.5,-1.25)  node at(1.7,-1.25){$-$} 
(1.5,-1.25) -- (1,-1.25) -- (1,-1.75) to[R=$R_1$] (1,-2.75) --(1,-3) node[ground]{}
(1,-1.5) to[R=$R_2$] (5.5,-1.5){}
(3.5,1) to[R=$r_o$] (5.5,1){}
(5.5,-1.5) to[R=$R_L$] (5.5,1){}
(3.5,1) --(3.5,1) to[V=$\mu V_i$] (3.5,-0.5) node[ground]{}
(5.5,-1.5) to[R=$R_M$] (5.5,-2.75) --(5.5,-3) node[ground]{}
(5.5,1) --(6,1) node at(6.3,1){$V_0$}
node at(1.8,-0.3) {$V_i$}
node at(0.6,-1.75){$+$}
node at(0.6,-3){$-$}
node at(0.6,-2.5){$V_f$}
node at(6,0.3){$I_o$}
[arrows=<-,line width=.4pt](5.7,0.2)--(5.7,0.5);
\end{circuitikz}
}
	\end{center}
\caption{}
\label{fig:ee18btech11048_ss_circuit}
\end{figure}


\item Draw the block diagram and equivalent ciruit for $H$.
\\
\renewcommand{\thefigure}{\theenumi.\arabic{figure}}


\begin{figure}[!ht]
	\begin{center}
		\resizebox{\columnwidth}{!}{\begin{circuitikz}[american]
\usetikzlibrary{positioning, fit, calc}
\draw (0,0)to [open,v=$V_f$]++(0,-2)to[short]++(2.5,0)
(0,0)to++(2.5,0);
\draw (4,-1)node[draw,minimum width=3cm,minimum height=3cm] (load) {H}(8,0);
\draw (5.5,0)--(8,0)
(5.5,-2)--(8,-2)
node at(6.7,-0.42){$I_o$}
[arrows=<-,line width=1pt](6.9,-0.2)--(6.4,-0.2);
;
\end{circuitikz}
}
	\end{center}
\caption{}
\label{fig:ee18btech11048pic3}
\end{figure}

\begin{figure}[!ht]
	\begin{center}
		\resizebox{\columnwidth}{!}{\usetikzlibrary{decorations.markings}
\begin{circuitikz}
\ctikzset{bipoles/length=1cm}

\draw 
(1,-1.5) to[R=$R_1$] (1,-2.75) --(1,-3) node[ground]{}
(1,-1.5) to[R=$R_2$,*-*] (5.5,-1.5){}
(5.5,-1.5)-- (5.5,-1){}
(5.5,-1.5) to[R=$R_M$] (5.5,-2.75) --(5.5,-3) node[ground]{};
\draw node at(0.6,-1.5){$V_f$}
node at(6,-1.2){$I_o$}
[arrows=<-,line width=.4pt](5.7,-1.4)--(5.7,-1);
\end{circuitikz}
}
	\end{center}
\caption{}
\label{fig:ee18btech11048h}
\end{figure}

\solution See Fig. \ref{fig:ee18btech11048pic3} and \ref{fig:ee18btech11048h}.
\item Find  $H$.
\\
\solution From Fig.  \ref{fig:ee18btech11048pic3} and \ref{fig:ee18btech11048h},

\begin{align}
H &= \frac{V_{f}}{I_{o}} 
\\
&= \frac{R_1R_M}{R_1+R_2+R_M}
\label{eq:ee18btech11048_H}
\end{align}
%
\item Find  $G$.
\\
\solution From Fig. \ref{fig:ee18btech11048_ss_circuit},
\begin{align}
G &= \frac{I_{o}}{V_{i}} \label{eq:ee18btech11048_G}\\
&= \mu
\end{align}
\item  Find $T$.
\\
\begin{figure}[!ht]
	\begin{center}
		\resizebox{\columnwidth}{!}{\input{./figs/ee18btech11048_block.tex}}
	\end{center}
\caption{}
\label{fig:block}
\end{figure}

\solution
\begin{align}
T &= \frac{G}{1+GH} \label{eq:ee18btech11048_T}
\\
&= \frac{\mu \brak{R_1+R_2+R_M}}{R_1+R_2+R_M+ \mu R_1R_M}
\\
 &\approx \frac{1}{H}  = \frac{R_1+R_2+R_M}{R_1R_M} 
\end{align}

\item Summarize your results in a table.
\\
\solution See Table \ref{table:ee18btech11048_ Input_Table}


\begin{table}[!ht]
\centering
\input{./tables/ee18btech11048_tables2.tex}
\caption{}
\label{table:ee18btech11048_ Input_Table}
\end{table}


\item Find $I_o$ for the parameters given in  Table  \ref{table:ee18btech11048_ Example_Table}.\\
\begin{table}[!ht]
\centering
\input{./tables/ee18btech11048_tables3.tex}
\caption{}
\label{table:ee18btech11048_ Example_Table}
\end{table}
\solution  The following code computes the value of $I_o$ using the fact that
\begin{align}
I_o &= \frac{V_s}{H}\\
\end{align}
\begin{lstlisting}
codes/ee18btech11048/ee18btech11048_fbc.py
\end{lstlisting}
On running this code value of $I_o$ is printed on terminal. The value obtained is 0.003 A.
\item Verify your result through spice.
\\
\solution The following readme file provides necessary instructions to simulate the circuit in spice.
\begin{lstlisting}
codes/ee18btech11048/spice/README
\end{lstlisting}

The following netlist simulates the given circuit.
\begin{lstlisting}
codes/ee18btech11048/spice/feedback.net
\end{lstlisting}
On running the spice simulations the $I_o$ value is printed on terminal.
The value printed is 0.003003266 A.

We observe that the value obtained using SPICE simulation is very close to the value obtained from the python code.

So the approximation for $T$ gives accurate results.
\end{enumerate}
