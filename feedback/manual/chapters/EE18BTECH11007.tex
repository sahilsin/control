\begin{enumerate}[label=\thesection.\arabic*.,ref=\thesection.\theenumi]
\numberwithin{equation}{enumi}

 \item Part of the circuit of the MC1553 Amplifier is shown in circuit1 in  Fig. \ref{fig:ee18btech11007_circuit1} with values of various parameters given in  Table \ref{table:ee18btech11007}.  Draw the equivalent block diagrams.

\renewcommand{\thefigure}{\theenumi.\arabic{figure}}
 \begin{figure}[!ht]
	\begin{center}
		
		\resizebox{\columnwidth}{!}{\begin{circuitikz}[american]
\draw (0,0) node[npn](npn1){Q1}
(npn1.B) -- ++(-2,0) to [open, v^>=${V}_s$,*-] ++(0,-1) to node[ground]{}++(0,-0.25)

(npn1.E)--++(0,-0.75) to [R,l_=$R_{E1}$] ++(0,-3) to node[ground]{}++(0,-0.25)
(npn1.C) -- ++(0,0.75)to [R,l_=$R_{C1}$] ++(0,2)coordinate(A) ++(3,0);
\draw (4,1.5) node[npn](npn2){Q2}
(npn1.C)-- ++(0,0.75) -- (npn2.B)
(npn2.E) to node[ground]{}++(0,-0.0001)
(npn2.C) -- ++(0,1.5)to [R,l_=$R_{C2}$]++(0,2)coordinate(B) ++(3,0);
\draw (8,3) node[npn](npn3){Q3}
(npn2.C) --++(0,0.75)-- (npn3.B)
(npn3.C) to [short,i_<=$I_c$]++(0,1)to [R,l_=$R_{C3}$]++(0,2)coordinate(C)
(npn3.E)to [short,i_=$I_o$]++(0,-3.75)coordinate(b) to [R,l_=$R_{E3}$] ++(0,-3) to node[ground]{}++(0,-0.5)
(npn1.E) ++(0,-0.75)coordinate(a) 

(b)to [R,l_=$R_F$](a)

(npn3.C) to [short,*-]++(1,0)
node[label={right:$V_o$}]{} ++(3,0)
(A) to [short,*-]++(0,0.01) node[label={right:$+V_{cc}$}]{} ++(0,0.25)
(B) to [short,*-]++(0,0.01) node[label={right:$+V_{cc}$}]{} ++(0,0.25)
(C) to [short,*-]++(0,0.01) node[label={right:$+V_{cc}$}]{} ++(0,0.25)
(npn1.E)++(0,-0.25) to [short,*-]++(0.5,0)
node[label={right:$V_f$}]{} ++(1.5,0)
(npn1.B)++(-0.75,0) to [short,*-]++(0,1)
node[label={right:$V_i$}]{} ++(2,0);
\draw (npn3.C)++(2.75,-2)node[label={right:$R_{out}$}]{}--++(0,1)--++(-1.5,0)[->];
\draw (npn3.E)++(-2,-1)node[label={above:$R_{of}$}]{}--++(1.75,0)[->];
\draw (npn1.B)++(-2.5,-2)node[label={left:$R_{if}$}]{}--++(0,1.5)--++(0.5,0)[->]
;\end{circuitikz}
}
	\end{center}
\caption{}
\label{fig:ee18btech11007_circuit1}
\end{figure}
 
\begin{table}[!ht]
\centering
\input{./tables/EE18BTECH11007/table1.tex}
\caption{parameters}
\label{table:ee18btech11007}
\end{table}
\solution  The block diagrams are  available in Figs. \ref{fig:ee18btech11007_block_diagram} and 
\ref{fig:ee18btech11007_trans-conductance_blockdiagram}. 
\begin{figure}[!ht]
	\begin{center}
		
		\resizebox{\columnwidth}{!}{
    \tikzstyle{block} = [draw, fill=white, rectangle, 
minimum height=2em, minimum width=3em]
\tikzstyle{sum} = [draw, fill=white, circle, radius=1mm, node distance=1.5cm]
\tikzstyle{input} = [coordinate]
\tikzstyle{output} = [coordinate]
\tikzstyle{pinstyle} = [pin edge={to-,thin,black}]

\begin{tikzpicture}[auto,scale=0.10mm, node distance=1cm,>=latex']
\node [input, name=input] {};
\node [sum, right of=input] (sum1) {$\sum$};
\node [input, right of=sum1] (B1) {};
\node [input, right of=B1] (B2) {};
\node [block, below of=B2] (B3) {$\frac{1}{s}$};
\node [block, above of=B2] (B4) {$s$};
\node [sum, right of=B2] (B5) {$\sum$};
\node [input, right of=B5] (B6) {};
\node [input, above of=B6] (A1) {};
\node [input, above of=A1] (A2) {};
\node [block, right of=B6] (B7) {$\frac{1}{s}$};
\node [input, right of=B7] (B8) {};
\node [input, below of=B8] (A3) {};
\node [input, below of=A3] (A4) {};
\node [output, right of=B8] (output) {};

\draw [->] (input)--node[pos=0.00]{$X(s)$} node[pos=0.99]{$+$}(sum1);
\draw[-] (sum1)--(B1);
\draw[->](B1) |- node[pos=0.99] {} (B4);
\draw[->](B1) |- node[pos=0.99] {} (B3);
\draw [->](B4) -| node[pos=0.99] {$+$} (B5);
\draw [->](B3) -| node[pos=0.99] {$+$} (B5);
\draw [->](B5)--(B7);
\draw [-](B7)--(B8);
\draw [->](B8)--node[pos=0.99] {$Y(s)$} (output);
\draw [-](B6) -- (A2);
\draw [->](A2) -| node[pos=0.99] {$-$} (sum1);
\draw [-](B8)--(A4);
\draw [->](A4) -| node[pos=0.99] {$-$} (sum1);

\end{tikzpicture}}
	\end{center}
\caption{block diagram}
\label{fig:ee18btech11007_block_diagram}
\end{figure}

\begin{figure}[!ht]
	\begin{center}
		
		\resizebox{\columnwidth}{!}{\begin{circuitikz}[american]
\usetikzlibrary{positioning, fit, calc}
\draw (0,0)to [voltage source,v=$V_s$]++(0,-2)
(0,0)to[R=$R_s$,i=$I_i$](6,0)to[R,v^>=${V}_i$,*-]++(0,-2)--++(-2,0)--++(0,-2)--++(4,0)to[controlled voltage source=$HI_o$]++(0,-2)--++(-6,0)--++(0,4)--++(-2,0)--(0,-2);
\draw (10,0) to [controlled current source=$GV_i$]++(0,-2)
(10,0)--(12,0)to[R=$R_o$]++(0,-2)--(14,-2)--(14,-4)--(12,-4)--(12,-6)--++(4,0)--++(0,4)--++(2,0)to[R=$R_L$]++(0,2)to [short,i=$I_o$](12,0)
(10,-2)--(12,-2)
(5,0)coordinate(left)
(5,-2)coordinate(bottoml)
(13,0)coordinate(right)
(13,-2)coordinate(bottomr)
node[fit=(left)(right)(bottoml)(bottomr),draw, dashed, label={amplifier circuit},inner sep=10pt] {}
(5,-4)coordinate(left1)
(5,-6)coordinate(bottoml1)
(13,-4)coordinate(right1)
(13,-6)coordinate(bottomr1)
node[fit=(left1)(right1)(bottoml1)(bottomr1),draw, dashed, label={feedback circuit},inner sep=10pt] {}
(5,-1)node[label={right:$R_i$}]{}
(2,-2)to [open, v^<=${V}_f$,*-]++(2,0)

;\end{circuitikz}
}
	\end{center}
\caption{Feedback Transconductance Amplifier}
\label{fig:ee18btech11007_trans-conductance_blockdiagram}
\end{figure}
\renewcommand{\thefigure}{\theenumi}

\item Draw the block diagram and equivalent circuit for $H$ for Fig. \ref{fig:ee18btech11007_trans-conductance_blockdiagram}.
\\
\solution Fig. \ref{fig:ee18btech11007_H_blockdiagram} gives the required block diagram
\renewcommand{\thefigure}{\theenumi.\arabic{figure}}
\begin{figure}[!ht]
	\begin{center}
		
		\resizebox{\columnwidth}{!}{\begin{circuitikz}[american]
\usetikzlibrary{positioning, fit, calc}
\draw (0,0)to [open,v=$V_f$]++(0,-2)to[short]++(6,0)
(0,0)to[short,i=$I_1$]++(6,0);
\draw (8,-1)node[draw,minimum width=4cm,minimum height=4cm] (load) {Feedback Network}(8,0)
(10,0)--++(6,0)
(10,-2)--(16,-2)to[current source,l_=$I_o$]++(0,2)
;
\end{circuitikz}
}
	\end{center}
\caption{Feedback circuit block diagram}
\label{fig:ee18btech11007_H_blockdiagram}
\end{figure}
\begin{align}
    H=\frac{V_f}{I_o}|_{I_{1}=0} 
\end{align}
%
and the equivalent $H$ circuit is available in Fig. \ref{fig:ee18btech11007_circuit2}.

\begin{figure}[!ht]
	\begin{center}
		
		\resizebox{\columnwidth}{!}{\begin{circuitikz}[american ]
\draw (0,0) to [R,l_=$R_{E1}$](0,-2) to node[ground]{}++(0,-0.25) ++(6,0)
(0,0) to [R,l_=$R_F$](3,0) to [R,l_=$R_{E2}$]++(0,-2)to node[ground]{}++(0,0)
(3,0)--(5,0)to [current source,l_=$-I_o$]++(0,-2) to node[ground]{}++(0,-0.5)
(0,0)--(-2,0) to [open, v^>=${V}_f$,*-] ++(0,-1) ++(6,0)
(0,0) 
;\end{circuitikz}
}
	\end{center}
\caption{H circuit}
\label{fig:ee18btech11007_circuit2}
\end{figure}
\renewcommand{\thefigure}{\theenumi}
%
\item Find the feedback Factor $H$
\\
\solution From Fig. \ref{fig:ee18btech11007_circuit2}, 
\begin{align}
    H&=\frac{V_f}{I_0}=\frac{R_{E1}R_{E2}}{R_{E2}+R_F+R_{E1}} 
%\\
%    &=\frac{100}{100+640+100}\times 100=11.9\ohm
\end{align}
\item Find $R_{11}$ and $R_{22}$  from Figs. \ref{fig:ee18btech11007_feedback_network} and \ref{fig:ee18btech11007_circuit2}
\begin{figure}[!ht]
	\begin{center}
		
		\resizebox{\columnwidth}{!}{\begin{circuitikz}[american]
\usetikzlibrary{positioning, fit, calc}
\draw (0,0)to [open]++(0,-2)to[short]++(3,0)
(0,0)to[short,i=$I_1$]++(3,0);
\draw (5,-1)node[draw,minimum width=4cm,minimum height=4cm] (load) {Feedback Network}(8,0)
(7,0)--(10,0)
(7,-2)--(10,-2)
(0,-1)node[label={right:$R_{11}$}]{}
(10,-1)node[label={right:$R_{22}$}]{} ++(4,0)

;
\end{circuitikz}
}
	\end{center}
\caption{feedback network}
\label{fig:ee18btech11007_feedback_network}
\end{figure}
\\
\solution
\begin{align}
    R_{11}=R_{E1}||(R_F+R_{E2})
\end{align}
\begin{align}
    R_{22}=R_{E2}||(R_F+R_{E1})
\end{align}
%
\item Draw the block diagram and equivalent circuit for $G$.
\\
\solution The required block diagram is available in Fig. \ref{fig:ee18btech11007_G_blockdiagram} 
\begin{figure}[!ht]
	\begin{center}
		
		\resizebox{\columnwidth}{!}{\begin{circuitikz}[american]
\usetikzlibrary{positioning, fit, calc}
\draw (0,0)to [voltage source,v=$V_i$]++(0,-2)to[R=$R_{11}$]++(6,0)
(0,0)to[R=$R_s$]++(6,0);
\draw (8,-1)node[draw,minimum width=4cm,minimum height=4cm] (load) {Basic Amplifier}(8,0)
(10,0)--++(6,0)to[R=$R_{L}$]++(0,-2)to [R=$R_{22}$,i<=$I_O$](10,-2)
;
\end{circuitikz}
}
	\end{center}
\caption{Amplifier circuit  block diagram}
\label{fig:ee18btech11007_G_blockdiagram}
\end{figure}
and the equivalent circuit in 
 \begin{figure}[!ht]
	\begin{center}
		
		\resizebox{\columnwidth}{!}{\begin{circuitikz}
\ctikzset{bipoles/length=1cm}
   \draw [R = $R_{6}$, o-] (3,3) to (6,3) ;
   \draw (6,3) to (6,2) node[ground] {};
   \draw [R = $R_{5}$, o-] (3,3) to (0,3) node[] {}
 node at (3,2.7){$V_{out}$}
 node at (0,2.7){$V_{in}$};
\end{circuitikz}}
	\end{center}
\caption{G circuit}
\label{fig:ee18btech11007_circuit3}
\end{figure}

%\begin{align}
%    G=\frac{I_o}{V_i}
%\end{align}
%$R_{11}$ and $R_{22}$ are  obtained from fig.\ref{fig:ee18btech11007_feedback_network}

%{\small
%\item draw the block diagram of a Feedback Trans-conductance Amplifier(series-series)
%\\
%\solution fig.\ref{fig:ee18btech11007_trans-conductance_blockdiagram} gives us the required block diagram 
%\begin{align}
%    T=\frac{I_o}{V_s}=\frac{G}{1+GH}
%\end{align}



\item Find $G$ 
\\
\solution 
To find $G=\frac{I_0}{V_i}$ we determine the gain of first stage,this is written by inspection as-
\begin{align}
    \frac{V_{c1}}{V_i}=\frac{-\alpha(R_{c1}||r_{\pi2})}{r_{e1}+(R_{E1}||(R_F+R_{E2}))}
\end{align}
%
Next, we determine the gain of the second stage,which can be written by inspection(noting that $V_{b2}=V_{c1}$)as
\begin{align}
    \frac{V_{c2}}{V_{c1}}=-g_{m2}{R_{c2}||(h_{fe}+1)[r_{e3}+(R_{E2}||(R_F+R_{E1}))]}
\end{align}
Finally,for the third stage we can write by inspection
\begin{align}
    \frac{I_0}{V_{c2}}=\frac{I_{e3}}{V_{b3}}=\frac{1}{r_{e3}+(R_{E2}||(R_F+R_{E1}))}
\end{align}

% finally Amplifier circuit is obtained shown in fig.\ref{fig:ee18btech11007_circuit3}
%\begin{figure}[!ht]
%	\begin{center}
%		
%		\resizebox{\columnwidth}{!}{\begin{circuitikz}{american}
\draw (0,0)node[npn](npn1){Q3} ++(5,0)
(npn1.B)to[R,l_=$R_{C2}$]++(-2,0) -- ++(0,-1)to node[ground]{}++(0,-0.25)
(npn1.E)to[R,l_=$R_{of}$]++(0,-1.5)to node[ground]{}++(0,-0.25)
(npn1.C)--++(0,1);
\draw (npn1.E)++(2,2)node[label={right:$R_{out}$}]{}--++(-1.8,0)--++(0,-0.5)[->]

;\end{circuitikz}
}
%	\end{center}
%\caption{circuit4}
%\label{fig:ee18btech11007_circuit4}
%\end{figure}
%using values from \ref{table:ee18btech11007}
%\begin{align}
%\frac{V_{c1}}{V_i}=-14.92V/V     
%\end{align}
%substituting ,results in 
%\begin{align}
%    \frac{V_{c2}}{V_{c1}}=-131.2 V/V
%\end{align}
%substituing values from \ref{table:ee18btech11007} gives
%\begin{align}
%    \frac{I_0}{V_{c2}}=10.6mA/V
%\end{align}
%combining the gains of the three stags results in
%\begin{align}
%G=\frac{I_0}{V_i}=-14.92\times-131.2\times10.6\times10^{-3}=20.7A/V    
%\end{align}
\item Find closed loop gain T and Voltage Gain $V_0/V_s$numerically.
\\ 
\solution
 \begin{align}
 \label{eq:EE18BTECH11007}
    T=\frac{I_0}{V_s}=\frac{G}{1+GH}=\frac{20.7}{1+20.7\times11.9}=83.7mA/V
\end{align}
% the voltage gain is found from 
%\begin{align}
%    \frac{V_0}{V_s}=\frac{-I_cR_{c3}}{V_s}\approx\frac{-I_0R_{C3}}{V_s}=-TR_{C3}
%    \
%\end{align}
%\begin{align}
%    =-83.7\times10^{-3}\times600=-50.2V/V
%\end{align}
\item Now assume Loop gain is large and find approximate expression for closed loop gain $T=\frac{I_o}{V_s}$
\\
\solution When GH $\gg$1,
\begin{align}
    T &\approx\frac{I_0}{V_s}\approx \frac{1}{H}
\\
    &=\frac{1}{11.9}=84mA/V
\end{align}
\begin{align}
 \frac{I_c}{V_s}\approx\frac{I_0}{V_s}=84 mA/V
 \end{align}
 which  we note is very close to the approximate value found in \eqref{eq:EE18BTECH11007} 
%\item Find $R_{in}$ and $R_{out}$ for circuit in  fig.\ref{fig:ee18btech11007_circuit1}
%\\
%\solution
%\begin{align}
%    R_{in} =R_{if}=R_i(1+GH)
%\end{align}
%where $R_i$ is the input resistance of the G circuit.The value of $R_i$ can be found from the circuit in fig.\ref{fig:ee18btech11007_circuit3} as follows:
%\begin{align}
%    R_i=(h_{fe}+1)(r_{e1}+(R_{E1}||(R_F+R_{E2})))=13.65K\Omega
%\end{align} 
%\begin{align}
%    R_{if}=13.65(1+20.7\times11.9)=3.38M\Omega
%\end{align}
%\begin{align}
%    R_{of}=R_o(1+GH)
%\end{align}
%where $R_o$ can be determined to be 
% \begin{align}
%    R_o=(R_{E2}||(R_F+R_{E1}))+r_{e3}+\frac{R_{C2}}{h_{fe}+1}
%\end{align}
%from values in Table \ref{table:ee18btech11007}, yields $R_o = 143.9 \Omega$. The output resistance $R_{of}$ of the feedback amplifier can now be found as
%\begin{align}
%    R_{of}=R_o(1+GH)=143.9(1+20.7\times11.9)=35.6K\Omega
%\end{align}
%$R_{out}$ is found by using circuit4 in fig.\ref{fig:ee18btech11007_circuit3}
%\begin{align}
%    R_{out}=r_{o3}+[R_{of}||(r_{\pi3}+R_{C2})](1+g_{m3}r_{o3}\frac{r_{\pi3}}{r_{\pi3}+R_{C2}})
%\end{align}
%\begin{align}
%=25+[35.6||(5.625)][1+160\times25\frac{0.625}{5.625}]=2.19M\Omega \end{align}
%
%thus $R_{out}$ is increased (from $r_{o3}$) but not by (1+GH)
\item Tabulate all your results.
\\
\solution See Table \ref{table:parameters2}.  
\begin{table}[!ht]
\centering
\input{./tables/EE18BTECH11007/table2.tex}
\caption{calculated parameters}
\label{table:parameters2}
\end{table}
%\item Represent this amplifier in  a control system Block Diagram
%\\
%\solution figure in  fig.\ref{fig:ee18btech11007_block_diagram} represents our control system
\item Write a code for doing calculations and verify the values obtained in \ref{table:parameters2} 
\\
\solution 
The following code does all the calculations of above equations to give parameters in
\ref{table:parameters2} 
\begin{lstlisting}
codes/ee18btech11007/circuit_calc.py
\end{lstlisting}



\end{enumerate}
