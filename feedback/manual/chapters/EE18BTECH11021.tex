\begin{enumerate}[label=\thesection.\arabic*.,ref=\thesection.\theenumi]
\numberwithin{equation}{enumi}
\item
Consider a Feedback Current Amplifier formed by cascading an Inverting Opamp $\mu$ with a MOSFET (NMOS) as shown in Fig. \ref{fig:ee18btech11021_Complete_Circuit}.
The output current is the Drain Current of the NMOS.
Assume that Opamp has an input resistance $R_{id}$, an Open Circuit Voltage Gain $\mu$, and an output resistance $r_{o1}$.  Express this as a control system.

\renewcommand{\thefigure}{\theenumi.\arabic{figure}}
\begin{figure}[!ht]
	\begin{center}
		\resizebox{\columnwidth}{!}{\input{figs/ee18btech11021/Complete_Circuit.tex}}
	\end{center}
\caption{Complete Circuit}
\label{fig:ee18btech11021_Complete_Circuit}
\end{figure}
\solution See Fig. \ref{fig:ee18btech11021_Block_Diagram}
%
\begin{figure}[!ht]
	\begin{center}
			\resizebox{\columnwidth}{!}{\input{figs/ee18btech11021/Block_Diagram.tex}}
	\end{center}
\caption{Block Diagram}
\label{fig:ee18btech11021_Block_Diagram}
\end{figure}
\renewcommand{\thefigure}{\theenumi}

\item Represent Fig. \ref{fig:ee18btech11021_Complete_Circuit} using a Small Signal Equivalent Model.

\solution See Fig. \ref{fig:ee18btech11021_Small_Signal_Model}

\begin{figure}[!ht]
	\begin{center}
		\resizebox{\columnwidth}{!}{\input{figs/ee18btech11021/Small_Signal_Model.tex}}
	\end{center}
\caption{Small Signal Model}
\label{fig:ee18btech11021_Small_Signal_Model}
\end{figure}

\item Find $G$.

\solution  From Fig. \ref{fig:ee18btech11021_Small_Signal_Model} we have the following equations
\begin{align}
V_g &= -\mu V_i
\\
\frac{V_i}{R_{id}} &= I_i
\\
I_s &= \frac{V_i}{R_s} + I_f + I_i
\\
I_f &= \frac{V_i-V_s}{R_2}
\\
\frac{V_g}{R_2} &= I_f + I_o
\\
I_o &= g_m \brak{V_g-V_s} + \frac{V_s}{r_{o2}}
\\
G &= \frac{I_o}{I_i}
\\
H &= \frac{I_f}{I_o}
\end{align}

%By replacing the Opamp with its equivalent model we can get the G circuit
%
%\begin{figure}[!ht]
%	\begin{center}
%		
%		\resizebox{\columnwidth}{!}{\begin{circuitikz}
\ctikzset{tripoles/mos style/arrows}
\draw
(0,0) node[nmos,](Q){}
(Q.center) node[right]{{$Q$}}
(Q.D) -- (0.75,0.75) to[R=$r_{02}$] (0.75,-0.75)-- (Q.S)
(Q.D) -- (0,1.5) -- (0.5,1.5) node[ground]{}
(Q.G) -- (-1,0) to[R=$r_{01}$, i = 0] (-4,0) 
(-4,-3.5) node[ground]{}
(-4,-3.5) to[V = $\mu V_i$] (-4,0){}
(Q.S) to[short, i = $I_{o}$] (0,-1.5) to[R=$R_1$] (0,-3.5) node[ground]{}
(0,-1.5) --(-1,-1.5) to[R=$R_2$] (-1,-3.5) node[ground]{}
;\end{circuitikz}
}
%	\end{center}
%\caption{Gain Circuit}
%\label{fig:ee18btech11021_Gain_Circuit}
%\end{figure}


\begin{align}
    R_{i}=R_{s}\|R_{i d}\|(R_{1}+R_{2})
\end{align}
%
where $R_i$ is the resistance seen by the current source $I_s$ and $R_{id}$ is the internal resistance of the OPAMP.
\begin{align}
    V_{i}&=I_{s} R_{i} 
\\
I_i &= I_s \frac{R_s\parallel \brak{R_1+R_2}}{R_s+R_{id}+R_1+R_2}
\end{align}
%
for small values of $I_f$.
\begin{align}
    I_{o}=-\mu V_{i} \frac{1}{1 / g_{m}+(R_{1}\|R_{2}\| r_{o 2})} \frac{r_{o 2}}{r_{o 2}+(R_{1} \| R_{2})}
\end{align}
\begin{align}
    G = \frac{I_{o}}{I_{i}}=-\mu \frac{R_{i}}{1 / g_{m}+(R_{1}\|R_{2}\| r_{o 2})} \frac{r_{o 2}}{r_{o 2}+(R_{1} \| R_{2})}
\end{align}
We use the approximation
\begin{align}
    1 / g_{m} \ll (R_{1}\|R_{2}\| r_{o 2})
\end{align}
This is because the $\frac{1}{g_{m}}$ is in order of few \ohm s but, $R_{1}$, $R_{2}$ and $r_{o2}$ are in order of k\ohm s 

\begin{align}
    G =-\mu \frac{R_{i}}{R_{1} \| R_{2}}
\end{align}
\begin{align}
    R_{o}=r_{o 2}+(R_{1} \| R_{2})+(g_{m} r_{o 2})(R_{1} \| R_{2})
\end{align}
\begin{align}
    \implies R_{o} \simeq g_{m} r_{o 2}\left(R_{1} \| R_{2}\right)
\end{align}

\item
Find expression for Loop Gain H

\solution
\begin{align}
    H = \frac{I_{f}}{I_{o}}=-\frac{R_{1}}{R_{1}+R_{2}}
\end{align}

\item
If loop gain is large, find approximate expression for closed loop gain $T$

\solution
Given,
\begin{align}
    GH \gg 1
\end{align}
\begin{align}
    T = \frac{G}{1+GH}\simeq \frac{1}{H}
\end{align}


\begin{align}
    T \simeq \frac{1}{H}=-\left(1+\frac{R_{2}}{R_{1}}\right)
\end{align}

\item
Give expressions for GH, $T$, $R_{if}$, $R_{in}$, $R_{of}$, $R_{out}$

\solution
\begin{align}
    GH=\mu \frac{R_{i}}{\frac{1}{g_{m}}+(R_{1}\|R_{2}\| r_{o 2})} \frac{r_{o 2}}{r_{o 2}+(R_{1} \| R_{2})} \frac{R_{1}}{R_{1}+R_{2}}
\end{align}

Once again, using the approximation,
\begin{align}
    \implies GH \simeq \mu \frac{R_{i}}{R_{1} \| R_{2}} \frac{R_{1}}{R_{1}+R_{2}}=\mu \frac{R_{i}}{R_{2}}
\end{align}

For Input Resistance,
\begin{align}
    R_{if}=R_{i} /(1+GH)
\end{align}
\begin{align}
    \implies \frac{1}{R_{i f}}=\frac{1}{R_{i}}+\frac{\mu}{R_{2}}
\end{align}
\begin{align}
    \implies R_{i f}=R_{i} \| \frac{R_{2}}{\mu}
\end{align}

Substituting the value of $R_{i}$,
\begin{align}
    R_{if}=R_{s}\|R_{id}\|(R_{1}+R_{2}) \| \frac{R_{2}}{\mu}
\end{align}

\begin{align}
    R_{if}=R_{s} \| R_{in}
\end{align}

\begin{align}
    \implies R_{in}=R_{i d}\|(R_{1}+R_{2})\| \frac{R_{2}}{\mu}
\end{align}
\begin{align}
    R_{in} \simeq \frac{R_{2}}{\mu}
\end{align}

For Output Resistance,
\begin{align}
    R_{of}=R_{o}(1+GH) \simeq GH R_{o}
\end{align}
\begin{align}
    R_{of} \simeq \mu (\frac{R_{i}}{R_{2}})(g_{m} r_{o 2})(R_{1} \| R_{2})
\end{align}
\begin{align}
    R_{out} = R_{of}=\mu \frac{R_{i}}{R_{1}+R_{2}}(g_{m} r_{o 2}) R_{1}
\end{align}


\item
Given the following values
\begin{table}[!ht]
\centering
\input{./tables/ee18btech11021/Input_Table.tex}
\caption{}
\label{table: Input_Table}
\end{table}

Find numerical value of $R_{i}$ and use it to find the value of G

\solution
Using the given numerical values on the previously obtained equations, we obtain:
\begin{align}
    R_{i}=\infty\|\infty\|(10+90)=100 k\ohm
\end{align}

\begin{align}
    G =-1000 \frac{100}{10 \| 90}=-11.11 \times 10^{3}
\end{align}

\item 
Check the validity of the approximation that we use to neglect $1/g_{m}$

\solution
\begin{align}
    1 / g_{m}=0.2 k\ohm \ll (10\|90\| 20)k\ohm = 6.2k\ohm
\end{align}
Hence, we can see that our approximation is valid

\item
Find the value of feedback gain H and open loop gain GH

\solution
\begin{align}
    H=-\frac{R_{1}}{R_{1}+R_{2}}=-\frac{10}{10+90}=-0.1
\end{align}

\begin{align}
    GH=1111 \gg 1
\end{align}

\item
Find the approximate value of closed loop gain T

\solution
\begin{align}
    T \simeq \frac{1}{H} = -\frac{1}{0.1} = -10
\end{align}

\item
Find the values of $R_{in}$ and $R_{out}$

\solution
\begin{align}
    R_{in}=\frac{R_{2}}{\mu}=\frac{90k\ohm}{1000}=90\ohm
\end{align}
\begin{align}
    R_{o} &=g_{m} r_{o 2}(R_{1} \| R_{2}) =5 \times 20(10 \| 90)=900k\ohm
\end{align}
\begin{align}
    R_{out}=(1+GH) R_{o}=1112 \times 900 \simeq 1000M\ohm
\end{align}

\begin{table}[!ht]
\centering
\input{./tables/ee18btech11021/Output_Table.tex}
\caption{}
\label{table: Output_Table}
\end{table}

\item
Verify the above calculations using a Python code.

\solution
\begin{lstlisting}
codes/ee18btech11021/ee18btech11021_calc.py
\end{lstlisting}

\end{enumerate}
