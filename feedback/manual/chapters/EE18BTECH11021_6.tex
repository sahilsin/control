\begin{enumerate}[label=\thesection.\arabic*.,ref=\thesection.\theenumi]
\numberwithin{equation}{enumi}
\item
Consider a Feedback Current Amplifier formed by cascading an Inverting Opamp $\mu$ with a MOSFET (NMOS).
The output current is the Drain Current of the NMOS.
Assume that Opamp has an input resistance $R_{id}$, an Open Circuit Voltage Gain $\mu$, and an output resistance $r_{o1}$

\renewcommand{\thefigure}{\theenumi.\arabic{figure}}
\begin{figure}[!ht]
	\begin{center}
		\resizebox{\columnwidth}{!}{\input{figs/ee18btech11021/Complete_Circuit.tex}}
	\end{center}
\caption{Complete Circuit}
\label{fig:ee18btech11021_Complete_Circuit}
\end{figure}
%
Identify the type of Feedback Circuit and draw its corresponding Block Diagram Representation

\solution
The Feedback Circuit is a Shunt-Series Feedback Current Amplifier.

See Figs. \ref{fig:ee18btech11021_Block_Diagram}
 and \ref{fig:ee18btech11021_Shunt_Series_Block_Diagram} for the block diagram representation.

\begin{figure}[!ht]
	\begin{center}
		\resizebox{\columnwidth}{!}{\begin{circuitikz}[american]
%\usetikzlibrary{positioning, fit, calc}
\draw (0,0) to[I = $I_{s}$] (0,2) -- (2,2) to[R=$R_s$,*-*] (2,0){}
(2,2) -- (5,2) {}
(7,1)node[draw,minimum width=4cm,minimum height=4cm] (load) {Gain Amplifier}{}
(7,-4)node[draw,minimum width=4cm,minimum height=4cm] (load) {Feedback Network}{}
(0,0)--(5,0)
(14,0) to[R=$R_L$,*-*] (14,2) to[short, i = $I_{o}$] (9,2)
(3,0) -- (3,-5) -- (5,-5){}
(4,2) -- (4,-3) -- (5,-3){}
(9,0) -- (10,0) -- (10,-3) -- (9,-3){}
(9,-5) -- (11,-5) -- (11,0) -- (14,0){}
;
\end{circuitikz}
}
	\end{center}
\caption{Shunt Series Amplifier Block Diagram}
\label{fig:ee18btech11021_Shunt_Series_Block_Diagram}
\end{figure}

\begin{figure}[!ht]
	\begin{center}
			\resizebox{\columnwidth}{!}{\input{figs/ee18btech11021/Block_Diagram.tex}}
	\end{center}
\caption{Block Diagram}
\label{fig:ee18btech11021_Block_Diagram}
\end{figure}
\renewcommand{\thefigure}{\theenumi}

\item
Represent the given circuit using a Small Signal Equivalent Model.

\solution See Fig. \ref{fig:ee18btech11021_Small_Signal_Model}.

To draw Small Signal Equivalent model,

\begin{itemize}
    \item Replace MOSFET with Current Source and Resistance in parallel with it.
    \item Replace Opamp with Voltage Source of the input Voltage multiplied by Gain
\end{itemize}

\begin{figure}[!ht]
	\begin{center}
		\resizebox{\columnwidth}{!}{\input{figs/ee18btech11021/Small_Signal_Model.tex}}
	\end{center}
\caption{Small Signal Model}
\label{fig:ee18btech11021_Small_Signal_Model}
\end{figure}

\item
Describe the resistances involved in the circuit

\solution  See Table \ref{table: Resistance_Table}

\begin{table}[!ht]
\centering
\input{./tables/ee18btech11021/Resistance_Table.tex}
\caption{}
\label{table: Resistance_Table}
\end{table}

\item
Draw the Block Diagram for $H$, with the corresponding Circuit.

\solution See Figs. \ref{fig:ee18btech11021_Feedback_Block}
and 
\ref{fig:ee18btech11021_Feedback_Circuit}
\begin{figure}[!ht]
	\begin{center}
		\resizebox{\columnwidth}{!}{\begin{circuitikz}[american]
\usetikzlibrary{positioning, fit, calc}
\draw (0,0) to[short, i = $I_{f}$] (0,2) -- (2,2)-- (5,2) {}
(7,1)node[draw,minimum width=4cm,minimum height=4cm] (load) {Feedback Network}{}
(0,0)--(5,0){}
(14,0) to[I = $I_{o}$] (14,2) -- (9,2)
(14,0) -- (9,0){}

;\end{circuitikz}}
	\end{center}
\caption{Feedback Block}
\label{fig:ee18btech11021_Feedback_Block}
\end{figure}

\begin{figure}[!ht]
	\begin{center}
		\resizebox{\columnwidth}{!}{ \begin{circuitikz}
\ctikzset{bipoles/length=1cm}

\draw 
(0,0) to[short,i=$I_f$] (1,0) to[R=$R_2$,o-*] (3,0) to[R=$R_1$] (3,-2) node[ground]{}
(0,0)--(0,-2) node[ground]{}
(3,0) -- (4,0){}
(4,-2) node[ground]{}
(4,-2) to[I=$I_o$] (4,0){}
;\end{circuitikz}}
	\end{center}
\caption{Feedback Circuit}
\label{fig:ee18btech11021_Feedback_Circuit}
\end{figure}
\item
Find the Feedback Gain H.
\solution In Fig. \ref{fig:ee18btech11021_Feedback_Circuit}, $I_{f}$ is the current in $R_{2}$
\begin{align}
    \implies H = \frac{I_{f}}{I_{o}}=-\frac{R_{1}}{R_{1}+R_{2}}
\end{align}

\item Find $R_{11}$ and $R_{22}$. 

\solution From  Fig. \ref{fig:ee18btech11021_Feedback_Circuit},
\begin{align}
    R_{11} = R_{1} + R_{2}
\end{align}
From Feedback Circuit, $R_{22}$ is resistance obtained by shorting Feedback Network

\begin{align}
    R_{22} = R_{1} \parallel R_{2}
\end{align}

\item Draw the Block Diagram for $G$, with the corresponding Circuit.

\solution See Figs. \ref{fig:ee18btech11021_Gain_Block}
and \ref{fig:ee18btech11021_Gain_Circuit}


\renewcommand{\thefigure}{\theenumi.\arabic{figure}}
\begin{figure}[!ht]
	\begin{center}
		\resizebox{\columnwidth}{!}{\begin{circuitikz}[american]
\usetikzlibrary{positioning, fit, calc}
\draw (0,0) to[I = $I_{i}$] (0,2) -- (2,2) to[R=$R_s$,*-*] (2,0){}
(2,2) -- (5,2) {}
(3,2) to[R=$R_{11}$,*-*] (3,0)
(7,1)node[draw,minimum width=4cm,minimum height=4cm] (load) {Gain Amplifier}{}
(0,0)--(5,0){}
(9,2) -- (14,2) to[R=$R_L$,*-*] (14,0)
(14,0) to [R=$R_{22}$,*-*] (9,0){}
(14,2) to[short,i=$I_o$] (9,2)

;\end{circuitikz}
}
	\end{center}
\caption{Gain Block}
\label{fig:ee18btech11021_Gain_Block}
\end{figure}
%
\begin{figure}[!ht]
	\begin{center}
		\resizebox{\columnwidth}{!}{\begin{circuitikz}
\ctikzset{tripoles/mos style/arrows}
\draw
(0,0) node[nmos,](Q){}
(Q.center) node[right]{{$Q$}}
(Q.D) -- (0.75,0.75) to[R=$r_{02}$] (0.75,-0.75)-- (Q.S)
(Q.D) -- (0,1.5) -- (0.5,1.5) node[ground]{}
(Q.G) -- (-1,0) to[R=$r_{01}$, i = 0] (-4,0) 
(-4,-3.5) node[ground]{}
(-4,-3.5) to[V = $\mu V_i$] (-4,0){}
(Q.S) to[short, i = $I_{o}$] (0,-1.5) to[R=$R_1$] (0,-3.5) node[ground]{}
(0,-1.5) --(-1,-1.5) to[R=$R_2$] (-1,-3.5) node[ground]{}
;\end{circuitikz}
}
	\end{center}
\caption{Gain Circuit}
\label{fig:ee18btech11021_Gain_Circuit}
\end{figure}
\renewcommand{\thefigure}{\theenumi}

\item Find the Gain $G$.

\solution In Fig. \ref{fig:ee18btech11021_Gain_Circuit},


\begin{align}
    R_{i}=R_{s}\parallel R_{i d}\parallel(R_{1}+R_{2})
\end{align}
\begin{align}
    V_{i}=I_{i} R_{i}
\end{align}
\begin{align}
    I_{o}=-\mu V_{i} \frac{1}{1 / g_{m}+(R_{1}\parallel R_{2}\parallel r_{o 2})} \frac{r_{o 2}}{r_{o 2}+(R_{1} \parallel R_{2})}
\end{align}
\begin{align}
    G = \frac{I_{o}}{I_{i}}=-\mu \frac{R_{i}}{1 / g_{m}+(R_{1}\parallel R_{2}\parallel r_{o 2})} \frac{r_{o 2}}{r_{o 2}+(R_{1} \parallel R_{2})}
\end{align}
We use the approximation
\begin{align}
    1 / g_{m} \ll (R_{1}\parallel R_{2}\parallel r_{o 2})
\end{align}
This is because the $\frac{1}{g_{m}}$ is in order of few \ohm s but, $R_{1}$, $R_{2}$ and $r_{o2}$ are in order of k\ohm s 

\begin{align}
    G =-\mu \frac{R_{i}}{R_{1} \parallel R_{2}}
\end{align}


\item
Calculate Loop Gain GH

\solution
\begin{align}
    GH=\mu \frac{R_{i}}{\frac{1}{g_{m}}+(R_{1}\parallel R_{2}\parallel r_{o 2})} \frac{r_{o 2}}{r_{o 2}+(R_{1} \parallel R_{2})} \frac{R_{1}}{R_{1}+R_{2}}
\end{align}

\begin{align}
    \implies GH \simeq \mu \frac{R_{i}}{R_{1} \parallel R_{2}} \frac{R_{1}}{R_{1}+R_{2}}=\mu \frac{R_{i}}{R_{2}}
\end{align}

\item
If loop gain is large, find approximate expression for closed loop gain $T$

\solution
Given,
\begin{align}
    GH \gg 1
\end{align}
\begin{align}
    T = \frac{G}{1+GH}\simeq \frac{1}{H}
\end{align}

\begin{align}
    T \simeq \frac{1}{H}=-\left(1+\frac{R_{2}}{R_{1}}\right)
\end{align}

\item
Give expressions for $R_{if}$, $R_{in}$

\begin{align}
    R_{if}=R_{i} /(1+GH)
\end{align}
\begin{align}
    \implies \frac{1}{R_{i f}}=\frac{1}{R_{i}}+\frac{\mu}{R_{2}}
\end{align}
\begin{align}
    \implies R_{i f}=R_{i} \parallel \frac{R_{2}}{\mu}
\end{align}

Substituting the value of $R_{i}$,
\begin{align}
    R_{if}=R_{s}\parallel R_{id}\parallel(R_{1}+R_{2}) \parallel \frac{R_{2}}{\mu}
\end{align}

\begin{align}
    R_{if}=R_{s} \parallel R_{in}
\end{align}

\begin{align}
    \implies R_{in}=R_{i d}\parallel(R_{1}+R_{2})\parallel \frac{R_{2}}{\mu}
\end{align}
\begin{align}
    R_{in} \simeq \frac{R_{2}}{\mu}
\end{align}

\item 
Give expressions for $R_{o}$, $R_{of}$, $R_{out}$

\solution
\begin{align}
    R_{o}=r_{o 2}+(R_{1} \parallel R_{2})+(g_{m} r_{o 2})(R_{1} \parallel R_{2})
\end{align}
\begin{align}
    \implies R_{o} \simeq g_{m} r_{o 2}\left(R_{1} \parallel R_{2}\right)
\end{align}
\begin{align}
    R_{of}=R_{o}(1+GH) \simeq GH R_{o}
\end{align}
\begin{align}
    R_{of} \simeq \mu (\frac{R_{i}}{R_{2}})(g_{m} r_{o 2})(R_{1} \parallel R_{2})
\end{align}
\begin{align}
    R_{out} = R_{of}=\mu \frac{R_{i}}{R_{1}+R_{2}}(g_{m} r_{o 2}) R_{1}
\end{align}


\item
Given the following values
\begin{table}[!ht]
\centering
\input{./tables/ee18btech11021/Input_Table.tex}
\caption{}
\label{table: Input_Table}
\end{table}

Find numerical value of $R_{i}$ and use it to find the value of G

\solution
Using the given numerical values on the previously obtained equations, we obtain:
\begin{align}
    R_{i}=\infty\parallel\infty\parallel(10+90)=100 k\ohm
\end{align}

\begin{align}
    G =-1000 \frac{100}{10 \parallel 90}=-11.11 \times 10^{3}
\end{align}

\item 
Check the validity of the approximation that we use to neglect $1/g_{m}$

\solution
\begin{align}
    1 / g_{m}=0.2 k\ohm \ll (10\parallel90\parallel 20)k\ohm = 6.2k\ohm
\end{align}
Hence, we can see that our approximation is valid

\item
Find the value of feedback gain H and open loop gain GH

\solution
\begin{align}
    H=-\frac{R_{1}}{R_{1}+R_{2}}=-\frac{10}{10+90}=-0.1
\end{align}

\begin{align}
    GH=1111 \gg 1
\end{align}

\item
Find the approximate value of closed loop gain T

\solution
\begin{align}
    T \simeq \frac{1}{H} = -\frac{1}{0.1} = -10
\end{align}

\item
Find the values of $R_{in}$ and $R_{out}$

\solution
\begin{align}
    R_{in}=\frac{R_{2}}{\mu}=\frac{90k\ohm}{1000}=90\ohm
\end{align}
\begin{align}
    R_{o} &=g_{m} r_{o 2}(R_{1} \parallel R_{2}) =5 \times 20(10 \parallel 90)=900k\ohm
\end{align}
\begin{align}
    R_{out}=(1+GH) R_{o}=1112 \times 900 \simeq 1000M\ohm
\end{align}

\begin{table}[!ht]
\centering
\input{./tables/ee18btech11021/Output_Table.tex}
\caption{}
\label{table: Output_Table}
\end{table}

\item
Verify the above calculations using a Python code.

\solution
\begin{lstlisting}
codes/ee18btech11021/ee18btech11021_calc.py
\end{lstlisting}

\end{enumerate}
