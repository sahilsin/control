\begin{enumerate}[label=\thesection.\arabic*.,ref=\thesection.\theenumi]
\numberwithin{equation}{enumi}

\item Fig. \label{fig:ee18btech11019_hart} shows a Hartley oscillator.

\begin{figure}[ht]
    \begin{center}
	    \resizebox{\columnwidth}{!}{\tikzset{
ifnode/.style={rectangle,dashed,draw=black, top color=white, inner sep=1em,minimum width=16cm, minimum height=5cm, text centered, fill=yellow}
}
\begin{circuitikz} [scale=2]
\node[ifnode, label=above:  Amplifier $G(s)$] (ifin) at (-2.7,0) {};
\node[ifnode, label=below:  Feedback $H(s)$] (ifin2) at (-2.8,-2.8) {};
\draw 
(0,0) node[op amp] (opamp) [scale=2] {}
(opamp.-) -- (-3,0.5) to [R=$R_1$] (-5,0.5) -- (-6,0.5) -- (-6,-2) to [L=$L_2$] (-4,-2)
-- (-1,-2) to[L=$L_1$] (1,-2) -- (1,0) {};
\node[draw,box] (A) at (1.5,-0.7) {$V_{out}$};
\node[draw,box] (B) at (-6.5,-3) {$V_{f}$};
\draw (-6,-2) -- (-6,-3)  to [C=$C$] (1,-3) -- (1,-2) 

(opamp.-) to[short,*-] ++(0,1) coordinate (leftC)
to[R=$R_2$] (leftC -| opamp.out)
to[short,-*] (opamp.out) to [short,-o] (1.5,0) to (1.5,-0.5) {}
(opamp.+) -- (-3,-0.5) -- (-5,-0.5) node[ground] [scale=2] {};
\draw (-2,-0.5) -- (-2,-2);
\draw (-6,-3) -- (-6.3,-3);


\end{circuitikz}
}
	\end{center}
\caption{Hartley oscillator}
\label{fig:ee18btech11019_hart}
\end{figure}
Draw the equivalent block diagram.
\solution  The block diagram is available in Fig. \label{fig:ee18btech11019_hart_block}
\begin{figure}[!ht]
    \begin{center}
		
		\resizebox{\columnwidth}{!}{\tikzset{
        amp/.style = {regular polygon, regular polygon sides=3,
              draw, fill=white, text width=1em,
              inner sep=1mm, outer sep=0mm,
              shape border rotate=-90},
        block/.style = {draw, rectangle,
            minimum height=1cm,
            minimum width=2cm},
        input/.style = {coordinate,node distance=1cm},
        output/.style = {coordinate,node distance=4cm},
        arrow/.style={draw, -latex,node distance=2cm},
        pinstyle/.style = {pin edge={latex-, black,node distance=2cm}},
        sum/.style = {draw, circle, node distance=1cm},
}

\begin{tikzpicture}[node distance=2.5cm,auto,>=latex']
  \node [input, name=input] {};
  \node [sum, right of=input] (sum) {};
  \node [amp, right of = sum] (block1) {$G$};
  \node [output, right of= block1] (output) {};
  \node [block, below of = block1] (block2) {$H$};
  \draw [->] (input) -- node {$V_{in}$} (sum);
  \draw [->] (sum) -- node {$V_{in} + HV_{out}$} (block1);
  \draw [->] (block1) -- node [name =y] {$V_{out}$} (output);
  \draw [->] (y) |- node [above,pos=0.79] {$V_{out}$} (block2) ;
  \draw [->] (block2) -| node  {$HV_{out}$} (sum) ;
\end{tikzpicture}
} %block diagram
	\end{center}
\caption{block diagram for oscillator}
\label{fig:ee18btech11019_hart_block}
\end{figure}
%
\item Find the gain of the oscillator. 
\\
\solution The control system in Fig. \ref{fig:ee18btech11019_hart_block} has positive feedback.  Hence, the gain is 

\begin{align}
    G = \frac{G(s)}{1-G(s)H(s)}
\label{eq:ee18btech11019_gain}
\end{align}
%
\item Find $G(s)$ and $H(s)$.
\\
\solution From figure \ref{fig:ee18btech11019_hart_block}
Oscillators gain can be given as follows:\\
\begin{align}
    A(V_{in} + BV_{out}) =V_{out}\\
    A(V_{in} = (1-AB)V_{out}\\
    \frac{V_{out}}{V_{in}} = \frac{A}{1 - AB}
\end{align}
%
resulting in \eqref{eq:ee18btech11019_gain}.\\



\item State the condition for sustained oscillations. Justify.

\solution Condition for sustained oscillation is given by\\
\begin{align}
    AB = 1
\end{align}
Along with, total phase gain o the circuit should be 0 or 2$\pi$\\
\textbf{Justification:} as, when $ AB =1 $, gain becomes infinity, and theoretically we can get output, without actually providing input\\
Total phase gain should be so, as we want our signal to be in phase after every loop traversal.\\


\item Find $A$ and $B$.

\solution Consider the below circuit fig \ref{fig:ee18btech11019_block2},its basic form of a LC oscillator.\\
%\begin{figure}[!ht]
%    \begin{center}
%		\resizebox{\columnwidth}{!}{\begin{circuitikz} 
(0,0) node[op amp] (opamp) [scale=2] {}
\draw
(0,0) -- (4,0) node[label = $V_{out}$]
  to [european resistor = $Z_2$]  (4,-4)   -- (0,-4)node[ground] {} to[european resistor = $Z_1$] (0,-2) 
  to[european resistor = $Z_3$] (0,0);
  \draw 
    (-2,0) node[op amp] (opamp) {} 
    (0,0) -- (opamp.out)
    (opamp.+)--(-4,-0.5)--(-4,-1)node[ground] {}
  (0,-2)--(-6,-2)--(-6,0.5) -- (opamp.-);
  

\end{circuitikz} 
}
%		
%	\end{center}
%\caption{block diagram for oscillator}
%\label{fig:ee18btech11019_block2}
%\end{figure}
The above figure \ref{fig:ee18btech11019_block2} can also be drawn as fig. \ref{fig:ee18btech11019_block3},when feedback is considered as load :\\
%\begin{figure}[!ht]
%    \begin{center}
%		\resizebox{\columnwidth}{!}{\begin{circuitikz} 

\draw
(0,0) -- (4,0) node[label = $V_{out}$]
  to [european resistor = $Z_2$]  (4,-4)   -- (0,-4)node[ground] {} to[european resistor = $Z_1$] (0,-2) 
  to[european resistor = $Z_3$] (0,0) to[american resistor,label=$R_o$] (-4,0) to[american voltage source ,label= -$A_vV_{in}$](-4,-1) node[ground] {}
  (0,-2)--(-6,-2)--(-6,0) node[label = $V_f$];
  

\end{circuitikz} 
}
%		
%	\end{center}
%\caption{block diagram for oscillator}
%\label{fig:ee18btech11019_block3}
%\end{figure}
We know that feedback gain is B, i.e, $\frac{V_0}{V_f} = B$\\
Applying voltage divider rule we get\\
From figure \ref{fig:ee18btech11019_block2}
\begin{align}
    B = \frac{Z_1}{Z_1 + Z_3}
\end{align}
From fig. \eqref{fig:ee18btech11019_block3}
\begin{align}
    A = \frac{V_o}{V_{in}} = \frac{A_vZ_L}{R_o + Z_L}\\
\end{align}    
    where,\\
    $A_v$ is the amplification factor of the opamp\\
    $v_{in}$ is the internal voltage in amplifier\\
    $Z_L$ is equivalent load across output
         
\begin{align}    
    Z_L = \frac{(Z_1 + Z_3)Z_2}{Z_1+Z_2+Z_3}
\end{align}


\item Find the frequency of oscillation using the condition that $AB = 1$.

\solution For any LC oscillator, 
Now,we know that $AB = 1$ for sustained oscillations, putting the the above terms in the equation\\
on solving,\\
\begin{align}    
    AB = \frac{Z_1Z_2A}{(Z_1+Z_2+Z_3)R_o+ Z_2(Z_1+Z_3)}
\end{align}    
\textbf{Hartley oscillator}:\\
The Hartley oscillator is one of the classical LC feedback circuits,i.e feedback is made of LC components.Below here 
\begin{align}
    Z_1 = SL_1 (inductor)\\
    Z_2 = SL_2 (inductor)\\
    Z_3 = \frac{1}{SC} (capacitor)
\end{align}

putting that in and equating $AB=1$ we get,

\begin{align}
1 = \frac{S^{2}L_1L_2A}{(SL_1+SL_2+\frac{1}{SC})R_o+ SL_2(SL_1+\frac{1}{SC})}\\
S^{2}L_1L_2A = (SL_1+SL_2+\frac{1}{SC})R_o+ SL_2(SL_1+\frac{1}{SC})
\end{align}

As we need, to find frequency, put S =jw
\begin{align}
    \omega^{2}L_1L_2A = j(\omega L_1 + \omega L_2 -\frac{1}{\omega C})R_o -\omega L_2(\omega L_1 + \frac{1}{\omega C})
\end{align}
To satisfy the above equation, equating imaginary term to Zero.
\begin{align}    
    \omega L_1 + \omega L_2  = \frac{1}{\omega C}\\
    \omega = \frac{1}{\sqrt{(L_1+L_2)(C)}}\\
    f = \frac{1}{2\pi \sqrt{(L_1+L_2)(C)}}
\end{align}
\begin{align}
    B = \frac{Z_1}{Z_1 + Z_3} = \frac{Z_1}{Z_2}\\
      = \frac{L_1}{L_2}
    \label{eq:ee18btech11019_B_gain}  
\end{align}
\begin{align}
    A =  \frac{L_2}{L_1} 
\label{eq:ee18btech11019_Amp_gain}
\end{align}
 
\item For Hartley oscillator frequency generated can be given as 
\begin{align}
    f = \frac{1}{2\pi\sqrt{(L_1 + L_2)C}}
    \label{eq:ee18btech11019_frequency}
\end{align}
Fig. \ref{fig:ee18btech11019_hart} shows a
Hartley oscillator built using opamp.\\

We can easily compare between \ref{fig:ee18btech11019_hart_block} and \ref{fig:ee18btech11019_hart}\\
We know that for an opamp A is given by:
\begin{align}
    A = \frac{R_2}{R_1}
\end{align}
Here,
\begin{align}
    A(S) = \frac{R_2}{R_1} = \frac{L_2}{L_1}
\end{align}
referring to \ref{eq:ee18btech11019_Amp_gain}\\
And,
\begin{align}
    B(s) = \frac{V_o}{V_f} = \frac{L_1}{L_2}
\end{align}
referring to \ref{eq:ee18btech11019_B_gain}\\
\newline
\item \textbf{Simulation:}\\
Taking the following values,and applying in \ref{eq:ee18btech11019_frequency} \\



\begin{tabular}{|c|c|}
\hline
Component & Value  \\
\hline
$R_1$         & 10K$\Omega$   \\
\hline
$R_2$         & 100K$\Omega$   \\
\hline
$R_3$         & $\sim$  \\
\hline
$L_1$         & $1 \mu H$     \\
\hline
$L_2$         & $1 \mu H$   \\
\hline
C         & 120 pF \\
\hline
\end{tabular}


We get f = 103 MHz\\
Feedback factor for Hartley given by:
\begin{align}
B =\frac{L_1}{L_2}= 1
\end{align}
W.K.T, $AB = 1$\\
$\therefore$ Minimum amplification Gain,A = 1\\
\end{enumerate}
