\begin{enumerate}[label=\thesection.\arabic*.,ref=\thesection.\theenumi]
\numberwithin{equation}{enumi}

\item The Block diagram of a system is illustrated in the figure shown, where $X(s)$ is the input and $Y(s)$ is the output. Draw the equivalent signal flow graph.
\renewcommand{\thefigure}{\theenumi.\arabic{figure}}

\begin{figure}[!ht]
    \begin{center}
		
		\resizebox{\columnwidth}{!}{
    \tikzstyle{block} = [draw, fill=white, rectangle, 
minimum height=2em, minimum width=3em]
\tikzstyle{sum} = [draw, fill=white, circle, radius=1mm, node distance=1.5cm]
\tikzstyle{input} = [coordinate]
\tikzstyle{output} = [coordinate]
\tikzstyle{pinstyle} = [pin edge={to-,thin,black}]

\begin{tikzpicture}[auto,scale=0.10mm, node distance=1cm,>=latex']
\node [input, name=input] {};
\node [sum, right of=input] (sum1) {$\sum$};
\node [input, right of=sum1] (B1) {};
\node [input, right of=B1] (B2) {};
\node [block, below of=B2] (B3) {$\frac{1}{s}$};
\node [block, above of=B2] (B4) {$s$};
\node [sum, right of=B2] (B5) {$\sum$};
\node [input, right of=B5] (B6) {};
\node [input, above of=B6] (A1) {};
\node [input, above of=A1] (A2) {};
\node [block, right of=B6] (B7) {$\frac{1}{s}$};
\node [input, right of=B7] (B8) {};
\node [input, below of=B8] (A3) {};
\node [input, below of=A3] (A4) {};
\node [output, right of=B8] (output) {};

\draw [->] (input)--node[pos=0.00]{$X(s)$} node[pos=0.99]{$+$}(sum1);
\draw[-] (sum1)--(B1);
\draw[->](B1) |- node[pos=0.99] {} (B4);
\draw[->](B1) |- node[pos=0.99] {} (B3);
\draw [->](B4) -| node[pos=0.99] {$+$} (B5);
\draw [->](B3) -| node[pos=0.99] {$+$} (B5);
\draw [->](B5)--(B7);
\draw [-](B7)--(B8);
\draw [->](B8)--node[pos=0.99] {$Y(s)$} (output);
\draw [-](B6) -- (A2);
\draw [->](A2) -| node[pos=0.99] {$-$} (sum1);
\draw [-](B8)--(A4);
\draw [->](A4) -| node[pos=0.99] {$-$} (sum1);

\end{tikzpicture}}
	\end{center}
\caption{Block Diagram}
\label{fig:ee18btech11003_block_diagram}
\end{figure}
\solution The signal flow graph of the block diagram in Fig. \ref{fig:ee18btech11003_block_diagram} is available in Fig. \ref{fig:ee18btech11003_signal_flow}
%
\begin{figure}[!ht]
\begin{center}
		
		\resizebox{\columnwidth}{!}{\input{./figs/ee18btech11003/signal_flow.tex}}
	\end{center}
\caption{Signal Flow Graph}
\label{fig:ee18btech11003_signal_flow}
\end{figure}
%
\renewcommand{\thefigure}{\theenumi}
\item Draw all the forward paths in Fig. \ref{fig:ee18btech11003_signal_flow}
and compute the respective gains.
\renewcommand{\thefigure}{\theenumi.\arabic{figure}}
\\
\solution The forward paths are available in Figs. \ref{fig:ee18btech11003_P1}
 and \ref{fig:ee18btech11003_P2}.  The respective gains are
\begin{align}
P_1&=s \brak{\frac{1}{s}}=1
\\
P_2&=(1/s)(1/s)=1/s^2
\end{align}
%
\begin{figure}[!ht]
\begin{center}
		
		\resizebox{\columnwidth}{!}{\input{./figs/ee18btech11003/P1.tex}}
	\end{center}
\caption{$P_1$}
\label{fig:ee18btech11003_P1}
\end{figure}
%
\begin{figure}[!ht]
\begin{center}
		
		\resizebox{\columnwidth}{!}{\input{./figs/ee18btech11003/P2.tex}}
	\end{center}
\caption{$P_2$}
\label{fig:ee18btech11003_P2}
\end{figure}
\renewcommand{\thefigure}{\theenumi}
%
\item Draw all the loops in Fig. \ref{fig:ee18btech11003_signal_flow} and calculate the respective gains.
\renewcommand{\thefigure}{\theenumi.\arabic{figure}}
\\
\solution The loops are available in Figs. \ref{fig:ee18btech11003_L1}-\ref{fig:ee18btech11003_L4}
and the corresponding gains are
%
\begin{align}
L_1&=(-1)(s)=-s
\\
L_2&=s\brak{\frac{1}{s}}\brak{-1}=-1
\\
L_3&=\brak{\frac{1}{s}}(-1)=-\frac{1}{s}
\\
L_4&=\brak{\frac{1}{s}}\brak{\frac{1}{s}}(-1)=-\frac{1}{s^2}
\end{align}

\begin{figure}[!ht]
\begin{center}
		
		\resizebox{\columnwidth}{!}{\input{./figs/ee18btech11003/L1.tex}}
	\end{center}
\caption{$L_1$}
\label{fig:ee18btech11003_L1}
\end{figure}



\begin{figure}[!ht]
\begin{center}
		
		\resizebox{\columnwidth}{!}{\input{./figs/ee18btech11003/L2.tex}}
	\end{center}
\caption{$L_2$}
\label{fig:ee18btech11003_L2}
\end{figure}



\begin{figure}[!ht]
\begin{center}
		
		\resizebox{\columnwidth}{!}{\input{./figs/ee18btech11003/L3.tex}}
	\end{center}
\caption{$L_3$}
\label{fig:ee18btech11003_L3}
\end{figure}



\begin{figure}[!ht]
\begin{center}
		
		\resizebox{\columnwidth}{!}{%L4
\begin{tikzpicture}
[
label revd/.is if=labrev,
amark/.style={
            decoration={             
                        markings,   
                        mark=at position {0.5} with { 
                                    \arrow{stealth},
                                    \iflabrev \node[above] {#1};\else \node[below] {#1};\fi
                        }
            },
            postaction={decorate}
},
terminal/.style 2 args={draw,circle,inner sep=2pt,label={#1:#2}},
]

%Place the nodes
\node[terminal={below left}{$N_1$}] (b) at (2cm,0) {};
\node[terminal={below left}{$N_2$}] (c) at (4cm,0) {};
\node[terminal={[xshift=-4mm]below right}{$N_3$}] (d) at (6cm,0) {};
\node[terminal={below right}{$N_4$}] (e) at (8cm,0) {};
\node[terminal={below left}{$N_5$}] (f) at (11cm,0) {};
%Draw the connections
\draw[amark=$ $] (b) to (c);
\draw[amark=$1/s$] (e) to (f);
\draw[amark=$ $] (d) to (e);
\draw[amark=$1/s$] (c) to[bend left=-45] (d);
\draw[amark=$-1$,label revd] (f) to[bend left=50] (b);
\end{tikzpicture}}
	\end{center}
\caption{$L_4$}
\label{fig:ee18btech11003_L4}
\end{figure}

\renewcommand{\thefigure}{\theenumi}

\item State Mason's Gain formula and explain the parameters through a table.
\\
\solution 
According to Mason's Gain Formula,
\begin{align}
T = \frac{Y(s)}{X(s)} 
\end{align}
\begin{align}
T = \frac{\sum_{i=1}^{N} P_i\Delta_i}{\Delta}
\end{align}
\item  Find the transfer function using Mason's Gain Formula.
\renewcommand{\thefigure}{\theenumi.\arabic{figure}}
%
\\
\solution 
%\begin{align}
% H(s)=\frac{Y(s)}{X(s)} 
%\end{align}

%Options -
% \begin{align}
% (A) - H(s)=\frac{s^2+1}{s^3+s^2+s+1}
% \end{align}
% \begin{align}
% (B) - H(s)=\frac{s^2+1}{s^3+2s^2+s+1}
% \end{align}
% \begin{align}
% (C) - H(s)=\frac{s^2+1}{s^2+s+1}
% \end{align}
% \begin{align}
% (D) - H(s)=\frac{s^2+1}{2s^2+1}
% \end{align}



Now, 

$P_i$ is the $i^{th}$ forward path.

$\Delta$ = 1 - (Sum of all individual loop gains)+(sum of gain products of all possible two non-touching loops)-(sum of gain products of all possible three non-touching loops)+...

$\Delta_i$ is  obtained from $\Delta$ by removing the loops which are touching the $i^{th}$ forward path.


$\Delta = 1-(L_1 + L_2 + L_3 + L_4)$

\begin{align}
L_1=(-1)(s)=-s
\end{align}

\begin{figure}[!ht]
\begin{center}
		
		\resizebox{\columnwidth}{!}{\input{./figs/ee18btech11003/L1.tex}}
	\end{center}
\caption{$L_1$}
\label{fig:sec_order}
\end{figure}


\begin{align}
L_2=\frac{s}{-s}=-1
\end{align}

\begin{figure}[!ht]
\begin{center}
		
		\resizebox{\columnwidth}{!}{\input{./figs/ee18btech11003/L2.tex}}
	\end{center}
\caption{$L_2$}
\label{fig:sec_order}
\end{figure}


\begin{align}
L_3=(\frac{1}{s})*(-1)=\frac{-1}{s}
\end{align}

\begin{figure}[!ht]
\begin{center}
		
		\resizebox{\columnwidth}{!}{\input{./figs/ee18btech11003/L3.tex}}
	\end{center}
\caption{$L_3$}
\label{fig:sec_order}
\end{figure}


\begin{align}
L_4=(\frac{1}{s})(\frac{1}{s})(-1)=\frac{-1}{s^2}
\end{align}

\begin{figure}[!ht]
\begin{center}
		
		\resizebox{\columnwidth}{!}{%L4
\begin{tikzpicture}
[
label revd/.is if=labrev,
amark/.style={
            decoration={             
                        markings,   
                        mark=at position {0.5} with { 
                                    \arrow{stealth},
                                    \iflabrev \node[above] {#1};\else \node[below] {#1};\fi
                        }
            },
            postaction={decorate}
},
terminal/.style 2 args={draw,circle,inner sep=2pt,label={#1:#2}},
]

%Place the nodes
\node[terminal={below left}{$N_1$}] (b) at (2cm,0) {};
\node[terminal={below left}{$N_2$}] (c) at (4cm,0) {};
\node[terminal={[xshift=-4mm]below right}{$N_3$}] (d) at (6cm,0) {};
\node[terminal={below right}{$N_4$}] (e) at (8cm,0) {};
\node[terminal={below left}{$N_5$}] (f) at (11cm,0) {};
%Draw the connections
\draw[amark=$ $] (b) to (c);
\draw[amark=$1/s$] (e) to (f);
\draw[amark=$ $] (d) to (e);
\draw[amark=$1/s$] (c) to[bend left=-45] (d);
\draw[amark=$-1$,label revd] (f) to[bend left=50] (b);
\end{tikzpicture}}
	\end{center}
\caption{$L_4$}
\label{fig:sec_order}
\end{figure}


$\Delta = 1-(-s-1-\frac{1}{s}-\frac{1}{s^2})$
$\Delta = \frac{s^3+2s^2+s+1}{s^2}$

\begin{figure}[!ht]
\begin{center}
		
		\resizebox{\columnwidth}{!}{%D1_D3
\begin{tikzpicture}
[
label revd/.is if=labrev,
amark/.style={
            decoration={             
                        markings,   
                        mark=at position {0.5} with { 
                                    \arrow{stealth},
                                    \iflabrev \node[above] {#1};\else \node[below] {#1};\fi
                        }
            },
            postaction={decorate}
},
terminal/.style 2 args={draw,circle,inner sep=2pt,label={#1:#2}},
]

%Place the nodes
\node[terminal={below left}{$N_1$}] (b) at (2cm,0) {};
\node[terminal={below right}{$N_4$}] (e) at (8cm,0) {};
%Draw the connections
\draw[amark=$-1$] (e) to[bend left=-45] (b);
\end{tikzpicture}}
	\end{center}
\caption{$\Delta_1$}
\label{fig:sec_order}
\end{figure}

After removing forward path from loop1 we will get Delta1

$\Delta_1 = 1$

\begin{figure}[!ht]
\begin{center}
		
		\resizebox{\columnwidth}{!}{%D2_D4
\begin{tikzpicture}
[
label revd/.is if=labrev,
amark/.style={
            decoration={             
                        markings,   
                        mark=at position {0.5} with { 
                                    \arrow{stealth},
                                    \iflabrev \node[above] {#1};\else \node[below] {#1};\fi
                        }
            },
            postaction={decorate}
},
terminal/.style 2 args={draw,circle,inner sep=2pt,label={#1:#2}},
]

%Place the nodes
\node[terminal={below left}{$N_1$}] (b) at (2cm,0) {};
\node[terminal={below left}{$N_5$}] (f) at (11cm,0) {};
%Draw the connections
\draw[amark=$-1$,label revd] (f) to[bend left=50] (b);
\end{tikzpicture}}
	\end{center}
\caption{$\Delta_2$}
\label{fig:sec_order}
\end{figure}

After removing forward path from loop2 we will get Delta2

$\Delta_2 = 1$

\begin{figure}[!ht]
\begin{center}
		
		\resizebox{\columnwidth}{!}{%D1_D3
\begin{tikzpicture}
[
label revd/.is if=labrev,
amark/.style={
            decoration={             
                        markings,   
                        mark=at position {0.5} with { 
                                    \arrow{stealth},
                                    \iflabrev \node[above] {#1};\else \node[below] {#1};\fi
                        }
            },
            postaction={decorate}
},
terminal/.style 2 args={draw,circle,inner sep=2pt,label={#1:#2}},
]

%Place the nodes
\node[terminal={below left}{$N_1$}] (b) at (2cm,0) {};
\node[terminal={below right}{$N_4$}] (e) at (8cm,0) {};
%Draw the connections
\draw[amark=$-1$] (e) to[bend left=-45] (b);
\end{tikzpicture}}
	\end{center}
\caption{$\Delta_3$}
\label{fig:sec_order}
\end{figure}

After removing forward path from loop3 we will get Delta4

$\Delta_3 = 1$

\begin{figure}[!ht]
\begin{center}
		
		\resizebox{\columnwidth}{!}{%D2_D4
\begin{tikzpicture}
[
label revd/.is if=labrev,
amark/.style={
            decoration={             
                        markings,   
                        mark=at position {0.5} with { 
                                    \arrow{stealth},
                                    \iflabrev \node[above] {#1};\else \node[below] {#1};\fi
                        }
            },
            postaction={decorate}
},
terminal/.style 2 args={draw,circle,inner sep=2pt,label={#1:#2}},
]

%Place the nodes
\node[terminal={below left}{$N_1$}] (b) at (2cm,0) {};
\node[terminal={below left}{$N_5$}] (f) at (11cm,0) {};
%Draw the connections
\draw[amark=$-1$,label revd] (f) to[bend left=50] (b);
\end{tikzpicture}}
	\end{center}
\caption{$\Delta_4$}
\label{fig:sec_order}
\end{figure}

After removing forward path from loop4 we will get Delta4

$\Delta_4 = 1$

Here, 
\begin{align}
T=\frac{\sum_{i=1}^{N}(P_i)(\Delta_i)}{\Delta}
\end{align}

\begin{align}
T=\frac{P_1 \Delta_1+P_2 \Delta_2+P_3 \Delta_3+P_4 \Delta_4}{\Delta}
\end{align}

\begin{align}
T=\frac{1*1 +(\frac{1}{s^2})*1 + 0*1 + 0*1 }{\frac{s^3+2s^2+s+1}{s^2}}
\end{align}

\begin{align}
H(s)=\frac{s^2+1}{s^3+2s^2+s+1}
\end{align}
\renewcommand{\thefigure}{\theenumi}

\end{enumerate}
