\begin{enumerate}[label=\thesubsection.\arabic*.,ref=\thesubsection.\theenumi]
\numberwithin{equation}{enumi}
\item 
The characteristic equation of linear time invariant system is given by
\begin{align} 
\nabla(s)=s^4+3s^3+3s^2+s+k=0
\end{align}
Find the condition for the system to be BIBO stable using the Routh Array.

\item \textbf{solution}
\begin{align}
\nabla(s)=s^4+3s^3+3s^2+s+k=0
\end{align}

\item Modify the Python code in Problem \ref{prob:ee18btech11005_python} to verify your solution by choosing two different values of $k$.

For a system to be stable all coefficients of characteristic equation should lie on left half of s-plane because if any of the pole is in right half of the s-plane then there will be a component which is exponentially increasing in output,causing system to be unstable.This can be verified by Routh Array Criterion.


The Routh hurwitz criterion:-


This criterion is based on arranging the coefficients of characteristic equation into an array called Routh array.If all the coefficients in the first row of routh array are of same algebric sign then the system is stable.


For any characteristic equation q(s),
\begin{multline}
q(s) = a_0s^n+a_1s^{n-1}+.....+a_{n-1}s+a_n = 0
\end{multline}
Routh array can be constructed as follows..,
 
\myvec{s^n\\s^{n-1}\\s^{n-2} \\ \vdots}
 \myvec{a_0 & a_2 & a_4 & \cdots \\
a_1 & a_3 & a_5 & \cdots  \\
b_1 & b_2 & b_3 & \cdots \\
\vdots & \vdots & \vdots & \ddots &\vdots 
 \cdots \\}
 \\
 where
 \begin{align}
 b_1 =\frac{ a_1a_2-a_0a_3}{a_1}  
 \\
 b_2 =\frac{ a_1a_4-a_0a_5}{a_1} 
 \\
 c_1=\frac{ b_1a_3-a_1b_2}{b_1} 
\\
 c_2=\frac{ b_1a_5-a_1b_3}{b_1}  
\end{align}
\bigskip
\begin{align}
\myvec{s^4\\s^3\\s^2\\s^1 \\ s^0}
\myvec{1 & 3 & k \\ 3 & 1 & 0\\  \frac{8}{3}& k & 0\\ \frac{\frac{8}{3}-3k}{\frac{8}{3}} & 0 & 0\\k & 0 & 0} 
\end{align}
Given system is stable if
\begin{align}
\dfrac{\dfrac{8}{3}-3k}{\dfrac{8}{3}}>0  ,  k>0
\end{align}
\begin{align}
{\dfrac{8}{3}-3k} >0
\end{align}
\begin{align}
3k<\dfrac{8}{3}
\end{align}
\begin{align}
(0<k<\dfrac{8}{9})
\end{align}
\bigskip
\\for example the zeros of polynomial $s^4+3s^3+3s^2+s+0.5=0$ are 
\begin{align}
s1=-0.08373+0.45773i
\end{align}
\begin{align}
s2=-0.08373-0.45773i
\end{align}
\begin{align}
s3=-1.41627+0.55075i
\end{align}
\begin{align}
s4=-1.41627-0.55075i
\end{align}

\end{enumerate}

