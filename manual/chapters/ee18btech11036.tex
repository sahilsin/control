\begin{enumerate}[label=\thesection.\arabic*.,ref=\thesection.\theenumi]
\numberwithin{equation}{enumi}

\item For the feedback system in Fig. \ref{fig:ee18btech11036},
%
\begin{align}
G(s) = \frac{1}{(s+1)(s+2)(s+3)}.
\label{eq:ee18btech11036_gain}
\end{align}
%
Find $k > 0$ for which the gain margin of system is exactly 0 dB and phase margin of system is exactly 0 degree.
\begin{figure}[!ht]
	\begin{center}
		
		\resizebox{\columnwidth}{!}{\begin{enumerate}[label=\thesection.\arabic*.,ref=\thesection.\theenumi]
\numberwithin{equation}{enumi}

\item For the feedback system in Fig. \ref{fig:ee18btech11036},
%
\begin{align}
G(s) = \frac{1}{(s+1)(s+2)(s+3)}.
\label{eq:ee18btech11036_gain}
\end{align}
%
Find $k > 0$ for which the gain margin of system is exactly 0 dB and phase margin of system is exactly 0 degree.
\begin{figure}[!ht]
	\begin{center}
		
		\resizebox{\columnwidth}{!}{\begin{enumerate}[label=\thesection.\arabic*.,ref=\thesection.\theenumi]
\numberwithin{equation}{enumi}

\item For the feedback system in Fig. \ref{fig:ee18btech11036},
%
\begin{align}
G(s) = \frac{1}{(s+1)(s+2)(s+3)}.
\label{eq:ee18btech11036_gain}
\end{align}
%
Find $k > 0$ for which the gain margin of system is exactly 0 dB and phase margin of system is exactly 0 degree.
\begin{figure}[!ht]
	\begin{center}
		
		\resizebox{\columnwidth}{!}{\begin{enumerate}[label=\thesection.\arabic*.,ref=\thesection.\theenumi]
\numberwithin{equation}{enumi}

\item For the feedback system in Fig. \ref{fig:ee18btech11036},
%
\begin{align}
G(s) = \frac{1}{(s+1)(s+2)(s+3)}.
\label{eq:ee18btech11036_gain}
\end{align}
%
Find $k > 0$ for which the gain margin of system is exactly 0 dB and phase margin of system is exactly 0 degree.
\begin{figure}[!ht]
	\begin{center}
		
		\resizebox{\columnwidth}{!}{\input{./figs/ee18btech11036.tex}}
	\end{center}
\caption{}
\label{fig:ee18btech11036}
\end{figure}
\\
\solution From the given information, system can be destabilized with a marginal increase in the gain. Hence the system is marginally stable.  The characteric equation for \eqref{eq:ee18btech11036_gain} is
\begin{align}
s^3+6s^2+11s^1+(6+k)
\end{align}
and the corresponding  Routh array is
\begin{align}
\mydet{s^3\\s^2\\s^1 \\ s^0}
\mydet{1 & 11 \\ 6 & (6+k) \\  \frac{66-(6+k)}{6} & 0\\ (6+k) & 0}
\end{align}
%
For the system to be  marginally stable,
%
\begin{align}
    \frac{66-(6+K)}{6}>0 \implies 
k=60
\end{align}
%
The following code
\begin{lstlisting}
codes/ee18btech11036.py
\end{lstlisting}
%
verifies that the system is marginally stable for $k = 60$.
\end{enumerate}





}
	\end{center}
\caption{}
\label{fig:ee18btech11036}
\end{figure}
\\
\solution From the given information, system can be destabilized with a marginal increase in the gain. Hence the system is marginally stable.  The characteric equation for \eqref{eq:ee18btech11036_gain} is
\begin{align}
s^3+6s^2+11s^1+(6+k)
\end{align}
and the corresponding  Routh array is
\begin{align}
\mydet{s^3\\s^2\\s^1 \\ s^0}
\mydet{1 & 11 \\ 6 & (6+k) \\  \frac{66-(6+k)}{6} & 0\\ (6+k) & 0}
\end{align}
%
For the system to be  marginally stable,
%
\begin{align}
    \frac{66-(6+K)}{6}>0 \implies 
k=60
\end{align}
%
The following code
\begin{lstlisting}
codes/ee18btech11036.py
\end{lstlisting}
%
verifies that the system is marginally stable for $k = 60$.
\end{enumerate}





}
	\end{center}
\caption{}
\label{fig:ee18btech11036}
\end{figure}
\\
\solution From the given information, system can be destabilized with a marginal increase in the gain. Hence the system is marginally stable.  The characteric equation for \eqref{eq:ee18btech11036_gain} is
\begin{align}
s^3+6s^2+11s^1+(6+k)
\end{align}
and the corresponding  Routh array is
\begin{align}
\mydet{s^3\\s^2\\s^1 \\ s^0}
\mydet{1 & 11 \\ 6 & (6+k) \\  \frac{66-(6+k)}{6} & 0\\ (6+k) & 0}
\end{align}
%
For the system to be  marginally stable,
%
\begin{align}
    \frac{66-(6+K)}{6}>0 \implies 
k=60
\end{align}
%
The following code
\begin{lstlisting}
codes/ee18btech11036.py
\end{lstlisting}
%
verifies that the system is marginally stable for $k = 60$.
\end{enumerate}





}
	\end{center}
\caption{}
\label{fig:ee18btech11036}
\end{figure}
\\
\solution From the given information, system can be destabilized with a marginal increase in the gain. Hence the system is marginally stable.  The characteric equation for \eqref{eq:ee18btech11036_gain} is
\begin{align}
s^3+6s^2+11s^1+(6+k)
\end{align}
and the corresponding  Routh array is
\begin{align}
\mydet{s^3\\s^2\\s^1 \\ s^0}
\mydet{1 & 11 \\ 6 & (6+k) \\  \frac{66-(6+k)}{6} & 0\\ (6+k) & 0}
\end{align}
%
For the system to be  marginally stable,
%
\begin{align}
    \frac{66-(6+K)}{6}>0 \implies 
k=60
\end{align}
%
The following code
\begin{lstlisting}
codes/ee18btech11036.py
\end{lstlisting}
%
verifies that the system is marginally stable for $k = 60$.
\end{enumerate}





